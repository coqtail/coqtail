
\documentclass[a4paper,11pt]{article}
 
% Import des extensions
\usepackage[T1]{fontenc}
\usepackage[utf8x]{inputenc}
\usepackage[french]{babel}

\begin{document}

Axiomatique de Pim : 
\begin{itemize}
\item Axiomes naturels (stdlib) sur les corps
\item Axiome de compatibilité du $<$
\item On dispose d'une fonction $up$ : R -> Z. $up$ est la partie entière + 1.
\item Pour toute fonction (f : nat -> bool) représentant l'écriture binaire d'un réel on peut 
associer un réel le plus proche (donc le réel de la représentation).
\end{itemize}


Problème : \\

La combinaison de $up$ et de l'axiome de completeness semble permettre de prouver des formules de $\Pi_1$ 
\footnote{Les propositions à un seul quantifieur universel}.\\

Démonstration :

Soit $P$ une proposition décidable : $\forall n, \{ P \ n \} + \{~P\ n\}$\\
On prend l'écriture binaire d : nat -> $\{ 0, 1 \}$ :\\
d n = 
0 si $(\exists k < n, ~P \ k)$ \\
1 sinon.\\

Soit x le réel défini par ce développement (axiome du completeness).\\

$up \ x = 2 <-> \forall k, P \ k$\\
\emph{Par exemple : \\
up (0,11111111...111) = up(1) = 2\\
up (0,11111111000... 0000) = 1\\
}
Donc :\\
Si P n'a pas de contre-exemple on a x = 0,111...111.. = 1 et donc up(x) = 2 \\ 
si P a un contre-exemple on a x = 0,1111...100.. et donc up(x) = 1. \\

Donc on a créé une procédure de décision pour P décidable.\\

Qed.

\end{document}
