\documentclass[11pt]{article}
\usepackage[T1]{fontenc}
\usepackage[utf8x]{inputenc}
\usepackage{amsmath, amssymb, amsthm}
\usepackage{csquotes}
\usepackage[french]{babel}
\usepackage{biblatex}
\usepackage{fullpage}

\bibliography{bibliography}

\newcommand{\coqtail}{\textsc{Coqtail}}

\begin{document}

\section{Brainstorming}

Blablabla les systèmes physiques avec du numérique dedans c'est l'avenir (bras de robot = du numérique mais aussi des temps de latence, des comportements continus = du sinusoidal, bordel !)

Donc important de modéliser le physique pour être capable de prouver des comportements (référence proco certifiés, calcul approché, norme IEEméfess)   

\begin{itemize}
  \item Trouver des abstractions puissantes pour décrire certains objets mathématiques (plus particulièrement tout ce qui touche au fait d'être $\mathcal{C}^n$) et pour les problèmes leur étant associés (équations différentielles).
  \begin{itemize}
    \item[$\Rightarrow$] Avantage : réutilisable massivement (the more general, the more reusable)
    \item[$\Rightarrow$] Donne des bases pour aborder des problèmes plus complexes : travail jamais perdu
  \end{itemize}

  \item Équas diffs super utiles dans toutes les disciplines ingénieuristiques... Tout est système continu et représentable par équa diff (asservissement...)

  \item Faire apparaitre le caractère purement algorithmique de certaines preuves (moué euh bon ?)
  \item Programmation certifiée
  \begin{itemize}
    \item[$\Rightarrow$] Vers de la mécanique certifiée ? (Oh my god, je me touche)
  \end{itemize}

  \item Vérification (dans longtemps) de problème pas forcément simple à vérifier (dont 3 personnes sont au courant de la preuve etc ?)
  \item Concours sur les 100 théorèmes (on va battre HOL il y a pas de raison)
\end{itemize}

%%%%%%%%%%%%%%%%%%

% TODO: On vire les passages sur l'importance mathématique de résultats en coq ? (je parle de ceux de coqtail)... Utilité dans la vie réelle ? Système physiques ? Faudrait dire plus clairement qu'on ne fait pas que de la branlette intélectuelle.

\section{Présentation Coqtail}

% TODO: Militantisme pro suppression des guillemets...

La correspondance de Curry--Howard établit le lien entre une preuve et un processus calculatoire : démontrer un théorème et construire une fonction qui, à partir des hypothèses, \emph{renvoie} la conclusion sont un seul et même problème.

Coq\cite{L:BC04} est un assistant de preuves formelles par ordinateur utilisant cette correspondance dans un sens comme dans l'autre. Il est ainsi possible :
\begin{itemize}
  \item d'énoncer des théorèmes mathématiques et de les prouver ;
  \item de définir des spécifications et de fournir une fonction les vérifiant.
\end{itemize}

Coq permet une construction interactive des preuves ainsi que la définition de tactiques permettant la résolution (totale ou partielle) de buts de manière automatique. Sa puissance n'est plus à démontrer étant donné qu'il a permis la formalisation de la preuve du théorème des quatres couleurs\cite{Gonthier07} ainsi que la création d'un compilateur certifié pour un important sous-ensemble de \texttt{C}\cite{compcert}.

% TODO :Problème de Bâle ?
Le projet \coqtail{} s'est, jusqu'à présent, attaché à fournir des bibliothèques permettant de faciliter les démonstrations en Coq. Le projet fait suite au projet COQUILLE développé au sein du module Projet Intégré du M1 d'Informatique Fondamentale.

Les développements actuels ont déjà permis de poser des bases solides au sein de divers domaines (suites réelles, complexes, analyse complexe, séries entières, etc.) et de démontrer des résultats dits élégants (formule de Stirling, indénombrabilité de $\mathbb{R}$, problème de Bâle) faisant l'objet d'une compétition entre les différents assistants de preuves\cite{Freek}.

% TODO:  euh le "à terme equa diff" c'est pas ce qu'on veut faire tout court? Cn c'est pas déjà largement défini(genre ça se fait en 5min)... Faudrait pas trop se foutre de leur gueule non?

Notre objectif est maintenant de mettre à profit les outils développés et la maturité acquise pour aborder des sujets plus fondamentaux et plus complexes. L'objectif principal de \coqtail{} sera donc de formaliser des notions de régularité telles que le fait d'être $\mathcal{C}^n$ (voire $\mathcal{C}^\infty$) ainsi que la notion d'équation différentielle. Ces formalisations seront accompagnées des outils nécessaires à leur manipulation, de leurs propriétés les plus basiques et de théorème plus ardus (notamment, à terme, les théorèmes de résolution d'équations différentielles).

% TODO:Retoucher le paragraphe suivant
Nos méthodes de travail sont directement héritées des mois passés à travailler sur les projets COQUILLE puis \coqtail{} : tout commence par l'analyse des résultats connus et la réécriture des grandes preuves permettant de dégager les résultats intermédiaires clefs. La définition des concepts est, ensuite, une étape cruciale qui nécessite un recul vis-à-vis du sujet à traiter et de ses enjeux. Une structure bien choisie permettra de faciliter les preuves tandis qu'un choix peu éclairé peu alourdir fortement la charge de travail à fournir.

% TODO on va pas leur apprendre ce qu'est un svn si? Re: Si ! 
Quels outils ?
Le projet \coqtail{} dispose déjà d'un site internet\cite{coqtail} et d'un système de versionnage de fichiers qui permet le travail collaboratif en parrallèle. Nous comptons continuer à utiliser ce mode de fonctionnement dans la mesure où il a fait ses preuves.

%% À VIRER ? {

Les théorèmes que nous prouverons pourront s'inscrire dans le challenge lancé par Freek (TODO 100 théorèmes machin machin ou alors remplacer ça par l'intérêt mathématiques) Notre travail pourra être réutilisé pour des développements plus complexes. Notre expérience (acquise lors du projet COQUILLE) nous permet de garantir l'obtention de résultats qui, même partiels, pourront être réutilisés.

%% }

\coqtail{} offre de réelles perspectives : au delà de fournir des preuves d'une certitude reconnue, %TODO
il permettra potentiellement d'aider à la modélisation de systèmes physique utilisant le numérique et de prouver leur comportement (un bras de robot par exemple). % Et mon cul c'est du poulet.

L'équipe comportera TODO membres et TODO MAIS BORDEL LA IL FAUT ETRE PRECIS SUR LE CALENDRIER ET ON SAIT MEME PAS CE QU'ON FAIT

\printbibliography

\end{document}