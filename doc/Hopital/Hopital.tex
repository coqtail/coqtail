\documentclass[11pt]{article}
\usepackage[T1]{fontenc}
\usepackage[utf8x]{inputenc}
\usepackage{amsmath, amssymb, amsthm}
\usepackage{csquotes}
\usepackage[french]{babel}
\usepackage{biblatex}
\usepackage{eurosym}
\usepackage{fullpage}
\usepackage{palatino}

%TODO aliais pour Hopital rule 

\begin{document}



\section{Presentation of the rule of l'Hopital}

L'Hôpital's rule is generally used to find the limit of fraction of undetermined form.
For example, given $lim_{x \rightarrow +\infty} f(x) = 0$ and $lim_{x \rightarrow +\infty} g(x) = 0$, l'hopital is used to determine $lim_{x \rightarrow +\infty} (f /g) (x)$ by successive derivations of f and g.
This theorem can be generalized to a lot of cases:  $f(x) \rightarrow +\infty$, $g(x) \rightarrow +\infty$, etc

A general statement of this rule is the following:\\
Given $(b1, b2) \in \{(+\infty, +\infty), (+\infty, - \infty), (- \infty, +\infty), (- \infty, -\infty) (0, 0)\}$ and $L \in \mathbb{R} \cup \{+\infty, -\infty\}$, 
$\forall a \in \mathbb{R} \cup {+\infty, -\infty}$, f and g are 2 continuous and derivable functions on a neighbourood of $a$ (g is not 0 on this neighborood, g' as well), 
if $lim_{x \rightarrow a} f (x) = b1$, $lim_{x \rightarrow a} g (x) = b2$ and $lim_{x \rightarrow a} (f'/g') (x) = L$  then  $lim_{x \rightarrow a} (f/g) (x) = L$.\\





\section{Implementation of the rule}

The general statement is not directy implementable in Coq as we want to use it in the Coq standard library of Reals. This library does not support using $+\infty$ and $-\infty$
as other Reals in what could be an extension of the $\mathbb{R}$ with new axioms for the infinities. So, we are obligated to define divergences in
$-\infty$ and $+\infty$.
As if it was not sufficient, we also need to distinguish $lim_{x \rightarrow a+}$ and $lim_{x \rightarrow a-}$ as these 2 limits can be sligthly different. For example, the inverse function
on 0.
We are also obligated to explicitly define the "neighborood".

So we need new definitions limits in $+\infty$ and $-\infty$. These definitions are exactly the one that can be found in a manual of maths. %TODO 

Definition: $lim_{x \rightarrow a+} f (x) =  +\infty$\\
$\forall a, b, m\in \mathbb{R}, a < b$, $m > 0 \rightarrow$ \\
$\exists \alpha \in \mathbb{R}, \alpha > 0 $ and \\
$\forall x, x \in ]a; b[ \rightarrow |x - a| < \alpha \rightarrow m < f(x)$
\\
Definition: $lim_{x \rightarrow b-} f (x) =  +\infty$\\
$\forall a, b, m\in \mathbb{R}, a < b$, $m > 0 \rightarrow$ \\
$\exists \alpha \in \mathbb{R}, \alpha > 0 $ and \\
$\forall x, x \in ]a; b[ \rightarrow |x - b| < \alpha \rightarrow m < f(x)$
\\
Definition: $lim_{x \rightarrow a+} f (x) =  -\infty$\\
$\forall a, b, m\in \mathbb{R}, a < b$, $m > 0 \rightarrow$ \\
$\exists \alpha \in \mathbb{R}, \alpha > 0 $ and \\
$\forall x, x \in ]a; b[ \rightarrow |x - a| < \alpha \rightarrow -m > f(x)$
\\
Definition: $lim_{x \rightarrow b-} f (x) =  -\infty$\\
$\forall a, b, m\in \mathbb{R}, a < b$, $m > 0 \rightarrow$ \\
$\exists \alpha \in \mathbb{R}, \alpha > 0 $ and \\
$\forall x, x \in ]a; b[ \rightarrow |x - b| < \alpha \rightarrow -m > f(x)$
\\


\section{Unnecessary hypotheses}

While proving this rule, we found out that some hypotheses were useless: 
In the case of $g \rightarrow +\infty$, we did not need the hypotheses $f \rightarrow +\infty$. The l'Hopital rule would be true even if 
the limit of f is finite which seems to be natural. But in Coq, this would avoid to the user to prove this statement. \\%TODO reverifier partout dans le code
The hypotheses "g' is not 0 on a neighborood" is sometimes as well not needed. For example when $g \rightarrow +\infty$, it is clear that there exists a region near infinity
where g is not going to be 0. We tried to avoid the user some work and avoid this hypothesis when possible.\\

\section{Cases implemented}

The file Hopital.v contains a proof of l'Hôpital's rule. 

These tables shows all the cases implemented yet:\\
\begin{tabular}{|c|c|c|c|}
  \hline
  when $x \rightarrow a+$ & L is finite & $L = +\infty$ & $L = -\infty$ \\
  \hline
  $g(x) \rightarrow 0$ & Hopital\_g0\_Lfin\_right & Hopital\_g0\_Lpinf\_right & Hopital\_g0\_Lninf\_right\\
  $g(x) \rightarrow +\infty$ & Hopital\_gpinf\_Lfin\_right & Hopital\_gpinf\_Lpinf\_right & Hopital\_gpinf\_Lninf\_right\\
  $g(x) \rightarrow -\infty$ & Hopital\_gninf\_Lfin\_right & Hopital\_gninf\_Lpinf\_right & Hopital\_gninf\_Lninf\_right\\
  \hline
\end{tabular}


\begin{tabular}{|c|c|c|c|}
  \hline
  when $x \rightarrow a-$ & L is finite & $L = +\infty$ & $L = -\infty$ \\
  \hline
  $g(x) \rightarrow 0$ & Hopital\_g0\_Lfin\_left & Hopital\_g0\_Lpinf\_left & Hopital\_g0\_Lninf\_left\\
  $g(x) \rightarrow +\infty$ & Hopital\_gpinf\_Lfin\_left & Hopital\_gpinf\_Lpinf\_left & Hopital\_gpinf\_Lninf\_left\\
  $g(x) \rightarrow -\infty$ & Hopital\_gninf\_Lfin\_left & Hopital\_gninf\_Lpinf\_left & Hopital\_gninf\_Lninf\_left\\
  \hline
\end{tabular}






\end{document}
