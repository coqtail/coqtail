
L'intégralité des membres du laboratoire junior a décroché un M2
en informatique théorique et poursuit ses études sous la forme
d'une thèse en France et/ou à l'étranger.

\paragraph{Allais Guillaume} commence une thèse à l'Université de
Strathclyde (Glasgow, Écosse) intitulée « Concevoir la précision
à l'aide des types dépendants » sous la direction de Conor McBride.
Le but de ce travail est de rendre accessible aux programmeurs
l'expressivité des types dépendants afin de leur permettre de
développer plus aisément des programmes certifiés corrects.

\paragraph{Marthe Bonamy} commence une thèse au LIRMM (Montpellier)
sur les « procédures de déchargements pour résoudre des problèmes
d'optimisation dans les graphes » sous la direction de Christophe
Paul, Benjamin Lévêque et Alexandre Pinlou.
Le but de ce travail est d'approfondir l'étude de techniques de
preuves (notamment utilisées pour démontrer le théorème des 4 couleurs).

\paragraph{Sylvain Dailler} commence une thèse entre
l'Université d'Orléans et la Kochi University of Technology (Japon)
sur la « compilation certifiée de langages concurrents de haut niveau »
sous la direction de Frédéric Dabrowski et Frédéric Loulergue.
Le but de ce travail est de développer en théorie et en pratique des
outils certifiés corrects traduisant les langages concurrents utilisés
par les programmateurs en instructions machines.

\paragraph{Jean-Marie Madiot} commence une thèse en co-tutelle entre
le LIP et l'Université de Bologne sous la direction de Daniel Hirschkoff
et Davide Sangiorgi. Il s'intéressera à l'étude formelle des programmes
concurrents et plus particulièrement au développement de techniques et
outils de preuve pour la certification des programmes s'éxécutant en
parallèle.

\paragraph{Pierre-Marie Pédrot} commence une thèse au sein de l'équipe
INRIA PiR2 (Paris-Rocquencourt) sous la direction d'Alexis Saurin et Hugo
Herbelin. Son travail consistera à formaliser un  calcul avec ressources
équipé de types dépendants conciliant ainsi deux branches jusqu'alors
orthogonales de la logique: la théorie des types de Martin-Löf et la logique
linéaire de Jean-Yves Girard.

