\documentclass[11pt]{article}
\usepackage[utf8x]{inputenc}
\usepackage[french]{babel}
\usepackage{amsmath, amsthm, amssymb, stmaryrd}
\usepackage{a4wide}
\usepackage{hyperref}

\newcommand{\coqtail}{\textsc{coqtail}}
\newcommand{\rsequence}{\texttt{Rsequence}}
\newcommand{\rpser}{\texttt{Rpser}}
\newcommand{\ring}{\texttt{ring}}
\newcommand{\field}{\texttt{field}}
\newcommand{\solve}{\texttt{solve\_diff\_equa}}

\DeclareMathOperator{\D}{\mathtt{An\_deriv}}
\DeclareMathOperator{\Dn}{\mathtt{An\_nth\_deriv}}
\DeclareMathOperator{\cvrw}{\mathtt{Cv\_radius\_weak}}
\DeclareMathOperator{\fcvr}{\mathtt{finite\_cv\_radius}}
\DeclareMathOperator{\icvr}{\mathtt{infinite\_cv\_radius}}

\DeclareMathOperator{\T}{\mathcal{T}}

\newcommand{\N}{\mathbb{N}}
\newcommand{\R}{\mathbb{R}}

\DeclareMathOperator{\Interp}{\mathtt{interp}}
\DeclareMathOperator{\Sem}{\left[\left| E_1 :=: E_2 \right|\right]}
\DeclareMathOperator{\SemR}{\Sem_{\R} \rho}
\DeclareMathOperator{\SemN}{\Sem_{\N} \rho}
\DeclareMathOperator{\IR}{\Interp_{\R}}
\DeclareMathOperator{\IN}{\Interp_{\N}}


\title{Rapport d'activité}
\author{Laboratoire junior \coqtail{}}
\date{Octobre 2011}

\begin{document}

\maketitle

\section{Rappel des buts}

\section{Contributions}

\subsection{Refonder les réels}

\subsection{Équations différentielles}

\subsubsection{Séries entières en coq}

La maîtrise par les membres du laboratoire junior de l'outil qu'est Coq a
considérablement évolué depuis la naissance du projet \coquille{}. Une première
partie du travail sur les équations différentielles a donc consisté à rendre les
développements sur les séries entières plus robustes en retravaillant les définitions
et en adaptant les preuves.

La formalisation des séries entières est maintenant très lourdement fondée sur les
bibliothèques décrivant les suites réelles ce qui a provoqué la diminution importante
de la longueur de certaines preuves. Ces changements de définitions ont par ailleurs
permis de prouver aisément des résultats plus avancés (produit de Cauchy de séries
entières par exemple).

\subsubsection{Résoudre des équations différentielles}

La formalisation des classes de fonctions $D^n$ et $C^n$ (sur $\R$ ou sur une boule
spécifique) et des fonctions dérivées $n$-ième a permis de définir des équations
différentielles. Leur manipulation est toutefois plutôt pesante en raison de la
complexité des termes.

Afin de permettre aux utilisateurs d'appliquer aisément des résultats sur les séries
entières, nous avons développé une tactique réflexive~\cite{reflection} permettant
de prouver des résultats sur les équations différentielles en prouvant des résultats
sur les coefficients des séries entières en jeu.
D'un terme complexe comprenant des dérivées d'ordres différents nous passons donc
à une expression plus simple qui pourra à son tour être déchargée en utilisant nos
bibliothèques sur les suites réelles.

\subsection{Graphes}

\section{Publications}

Le travail effectué au sein du laboratoire \coqtail{} a donné lieu à la soumission
de trois articles courts (\textit{extended abstract}) afin de présenter les résultats
au cours de différents ateliers de travail (\textit{workshop}) associés à des
conférences internationales. Tous les articles courts mentionnés ici sont disponibles
en appendices.

\subsection{Coq with power series - THedu}

THedu est un groupe de travail dont un des axes d'intérêt est \textit{Consistent
Mathematical Content Representation}. Notre article visait à mettre en avant la
nécessité absolue de prendre le temps de formaliser les notions abstraites nécessaires
à l'explicitation de la nature des objets que l'on souhaite définir.

Pour cela nous avons présenter la formalisation des séries entières effectuées au
sein du laboratoire et les conséquences pratiques :
\begin{itemize}
 \item définitions aisée de la fonction exponentielle et des fonctions
 trigonométriques usuelles,
 \item obtention de propriétés de ces fonctions comme corrollaires triviaux
 de résultats sur les séries entières (charactère $C^{\infty}$, lien entre les
 différentes fonctions, etc.),
 \item possibilité d'utiliser une tactique générique travaillant sur les séries
 entières lorsqu'on évoque ces fonctions
\end{itemize}

\subsection{Constructive axiomatic for the real numbers - Coq Workshop}

\subsection{Using reflection to solve some differential equations - Coq Workshop}

À l'inverse du public de THedu, le public du Coq Workshop est composé de chercheurs
habitués à utiliser Coq ce qui nous a permis de rentrer plus dans les détails de la
construction (élégante !) de la tactique par reflection.

Une potentielle amélioration de la robustesse du système de réification~\cite{adhoc}
du but courant nous a été suggérée à l'issue de la présentation.

\section{Devenir des membres de l'équipe}


L'intégralité des membres du laboratoire junior a décroché un M2
en informatique théorique et poursuit ses études sous la forme
d'une thèse en France et/ou à l'étranger.

\paragraph{Allais Guillaume} commence une thèse à l'Université de
Strathclyde (Glasgow, Écosse) intitulée « Concevoir la précision
à l'aide des types dépendants » sous la direction de Conor McBride.
Le but de ce travail est de rendre accessible aux programmeurs
l'expressivité des types dépendants afin de leur permettre de
développer plus aisément des programmes certifiés corrects.

\paragraph{Marthe Bonamy} commence une thèse au LIRMM (Montpellier)
sur les « procédures de déchargements pour résoudre des problèmes
d'optimisation dans les graphes » sous la direction de Christophe
Paul, Benjamin Lévêque et Alexandre Pinlou.
Le but de ce travail est d'approfondir l'étude de ces techniques de
preuves (notamment utilisées pour démontrer le théorème des 4 couleurs).

\paragraph{Sylvain Dailler} commence une thèse entre
l'Université d'Orléans et la Kochi University of Technology (Japon)
sur la « compilation certifiée de langages concurrents de haut niveau »
sous la direction de Frédéric Dabrowski et Frédéric Loulergue.
Le but de ce travail est de développer en théorie et en pratique des
outils certifiés corrects traduisant les langages concurrents utilisés
par les programmateurs en instructions machines.

\paragraph{Jean-Marie Madiot} commence une thèse en co-tutelle entre
le LIP et l'Université de Bologne sous la direction de Daniel Hirschkoff
et Davide Sangiorgi. Il s'intéressera à l'étude formelle des programmes
concurrents et plus particulièrement au développement de techniques et
outils de preuve pour la certification des programmes s'éxécutant en
parallèle.

\paragraph{Pierre-Marie Pédrot} commence une thèse au sein de l'équipe
INRIA PiR2 (Paris -- Rocquencourt) sous la direction d'Alexis Saurin et Hugo
Herbelin. Son travail consistera à formaliser un  calcul avec ressources
équipé de types dépendants conciliant ainsi deux branches jusqu'alors
orthogonales de la logique: la théorie des types de Martin-Löf et la logique
linéaire de Jean-Yves Girard.



\bibliographystyle{amsplain}
\bibliography{biblio}

\end{document}

