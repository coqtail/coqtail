\documentclass[11pt]{article}
\usepackage[utf8x]{inputenc}
\usepackage[french]{babel}
\usepackage{amsmath, amsthm, amssymb, stmaryrd}
\usepackage{a4wide}
\usepackage{hyperref}

\newcommand{\coqtail}{\textsc{coqtail}}
\newcommand{\rsequence}{\texttt{Rsequence}}
\newcommand{\rpser}{\texttt{Rpser}}
\newcommand{\ring}{\texttt{ring}}
\newcommand{\field}{\texttt{field}}
\newcommand{\solve}{\texttt{solve\_diff\_equa}}

\DeclareMathOperator{\Dn}{\mathtt{An\_nth\_deriv}}
\DeclareMathOperator{\cvrw}{\mathtt{Cv\_radius\_weak}}
\DeclareMathOperator{\fcvr}{\mathtt{finite\_cv\_radius}}
\DeclareMathOperator{\icvr}{\mathtt{infinite\_cv\_radius}}

\DeclareMathOperator{\T}{\mathcal{T}}

\newcommand{\N}{\mathbb{N}}
\newcommand{\R}{\mathbb{R}}

\DeclareMathOperator{\Interp}{\mathtt{interp}}
\DeclareMathOperator{\Sem}{\left[\left| E_1 :=: E_2 \right|\right]}
\DeclareMathOperator{\SemR}{\Sem_{\R} \rho}
\DeclareMathOperator{\SemN}{\Sem_{\N} \rho}
\DeclareMathOperator{\IR}{\Interp_{\R}}
\DeclareMathOperator{\IN}{\Interp_{\N}}


\title{Rapport d'activité}
\author{Laboratoire junior \coqtail{}}
\date{Octobre 2011}

\begin{document}

\maketitle

\section{Rappel des buts}
% TODO reprendre... pomper et de grosses omissions
\subsection{Conjoncture}
% TODO pomper de la proposal
La correspondance de Curry-Howard établit le lien entre une preuve et un processus calculatoire : 
démontrer un théorème et construire une fonction qui, à partir des hypothèses, renvoie la conclusion 
sont un seul et même problème.\\

Coq est un assistant de preuves formelles par ordinateur utilisant cette correspondance dans un sens comme dans l'autre. Il est ainsi possible :
d'énoncer des théorèmes mathématiques et de les prouver ;
de définir des spécifications et de fournir une fonction les vérifiant.\\

Coq permet une construction interactive des preuves ainsi que la définition de tactiques permettant la résolution (totale ou partielle) de buts de manière automatique. Sa puissance n'est plus à démontrer étant donné qu'il a permis la formalisation de la preuve du théorème des quatre couleurs ainsi que la création d'un compilateur certifié pour un important sous-ensemble de C.

\subsection{Héritage de \coquille{}}

Le projet \coqtail{} (Coq Theorems, Abstractions and Implementations, Licence-level) s'était, au début, attaché à fournir des outils (formalisations d'objets et résultats théoriques attachés) permettant d'aborder simplement des problèmes de niveau Licence en Coq. Il fait suite à \coquille{} développé au sein du module Projet Intégré du M1 d'Informatique Fondamentale.

\subsection{Objectifs}

Notre objectif était de mettre à profit les outils développés et la maturité acquise lors du projet \coquille{} pour aborder des sujets plus fondamentaux et plus complexes. Dans ce cadre, nos efforts ont été orientés vers la formalisation et l'étude des équations différentielles d'une part et des graphes d'autre part. Ces formalisations, bien entendu, accompagnées des outils nécessaires à leur manipulation, de leurs propriétés les plus basiques et de théorèmes plus ardus (notamment les théorèmes de résolution d'équations différentielles).\\

Nous voulions également refonder les réels en Coq. Nous proposons notre propre formalisation des réels qui nous permettra de calculer avec des réels tout en conservant la possibilité de prouver des théorèmes classiques. Cette construction devait servir de base aux développements d'analyse réelle mais être également reliée à l'axiomatique de la bibliothèque standard afin de faire en sorte que nos résultats soient réutilisables par le plus grand nombre.

\section{Contributions}

\subsection{Refonder les réels}

\subsubsection{Des objets infinis}

Les nombres réels posent un problème récurrent dans le domaine de l'informatique. Ce sont des objets mathématiques infinis et les représenter dans le monde fini de l'informatique, et \emph{a fortiori} de Coq, se révèle d'une grande difficulté. Dans les programmes usuels et les logiciels de calcul scientifique, les réels sont réduits à une approximation, qu'on appelle communément des flottants. À l'inverse pour énoncer et prouver des propriétés mathématiques sur les réels, ces approximations ne suffisent plus : on a besoin d'une représentation exacte des nombres réels.

\subsubsection{Constructivisme}

Ces nombres réels existent déjà dans Coq, mais leur représentation relève de la logique classique et les utiliser revient à abandonner la partie constructive de Coq. L'aspect constructif de Coq est un aspect important qui est intimement rélié à la correspondance de Curry-Howard. Il permet d'effectuer des calculs à l'intérieur de Coq, ce qui fait de lui un système à la fois de preuves mathématiques et de calculs informatiques. Cet aspect a fait ses preuves dans la preuve du théorème des quatre couleurs par Georges Gonthier et Benjamin Werner en 2005 : ce théorème implique qu'on peut colorier chaque région d'une carte de sorte qu'aucune région n'en touche une autre de même couleur. La preuve de ce théorème est \emph{constructive} et de ce fait, on peut en déduire un programme qui, étant donnée une carte, colorie cette carte en respectant cette condition.

\subsubsection{Problématique}

On souhaite donc conserver cet aspect constructif pour la définition des nombres réels. Cependant, la logique classique intervient aussi dans les démonstrations mathématiques usuelles. On cherche donc une définition des réels qui permet les preuves constructives et le calcul sur les nombres réels d'une part, et qui autorise l'utilisation d'une logique classique d'autre part.

\subsubsection{Proposition}

On propose alors une définition qui n'est plus basée sur la coupure de Dedekind, qui est essentiellement non constructive, mais sur les suites de Cauchy. Si on ne s'intéresse pas au constructivisme, ces deux définitions sont équivalentes. En revanche quand on s'y intéresse, seules les suites de Cauchy permettent un calcul des nombres réels. Un autre problème se pose, quand on divise par un nombre : en mathématique, on s'assure que le nombre est différent de zéro pour continuer. En construisant les nombres réels, on exige une preuve que ce nombre est non nul d'une manière qui autorise le calcul de l'inverse.

La définition des réels contient maintenant des techniques de séparation des preuves constructives et des preuves classiques. Elles utilisent un mécanisme basé sur les monades, un objet mathématique permettant l'utilisation de certains axiomes dans certains théorèmes ou programmes sans mettre en danger la cohérence des autres parties.

\subsection{Équations différentielles}

\subsubsection{Séries entières en coq}

La maîtrise par les membres du laboratoire junior de l'outil qu'est Coq a
considérablement évolué depuis la naissance du projet \coquille{}. Une première
partie du travail sur les équations différentielles a donc consisté à rendre les
développements sur les séries entières plus robustes en retravaillant les définitions
et en adaptant les preuves.

La formalisation des séries entières est maintenant très lourdement fondée sur les
bibliothèques décrivant les suites réelles ce qui a provoqué la diminution importante
de la longueur de certaines preuves. Ces changements de définitions ont par ailleurs
permis de prouver aisément des résultats plus avancés (produit de Cauchy de séries
entières par exemple).

\subsubsection{Résoudre des équations différentielles}

La formalisation des classes de fonctions $D^n$ et $C^n$ (sur $\R$ ou sur une boule
spécifique) et des fonctions dérivées $n$-ième a permis de définir des équations
différentielles. Leur manipulation est toutefois plutôt pesante en raison de la
complexité des termes.

Afin de permettre aux utilisateurs d'appliquer aisément des résultats sur les séries
entières, nous avons développé une tactique réflexive~\cite{reflection} permettant
de prouver des résultats sur les équations différentielles en prouvant des résultats
sur les coefficients des séries entières en jeu.
D'un terme complexe comprenant des dérivées d'ordres différents nous passons donc
à une expression plus simple qui pourra à son tour être déchargée en utilisant nos
bibliothèques sur les suites réelles.

\subsection{Continuité des résultats de \coquille{}}

Dans la continuité du travail réalisé dans le projet \coquille{}, nous avons continué 
le travail de formalisations de théorèmes usuels en analyse réelle. En particulier, 
nous avons prouvé la règle de l'Hopital dans de nombreux cas intéressants. Ce résultat 
permet d'établir des limites de fonctions très simplement. Nous avons également  
prouvé le théorème de Cauchy dans ces versions les plus simples. Cela devrait permettre 
de faciliter le travail sur les équations différentielles. 

\subsection{Graphes}

\section{Publications}

Le travail effectué au sein du laboratoire \coqtail{} a donné lieu à la soumission
de trois articles courts (\textit{extended abstract}) afin de présenter les résultats
au cours de différents ateliers de travail (\textit{workshop}) associés à des
conférences internationales. Tous les articles courts mentionnés ici sont disponibles
en appendices.

\subsection{Coq with power series - THedu}

THedu est un groupe de travail dont un des axes d'intérêt est \textit{Consistent
Mathematical Content Representation}. Notre article visait à mettre en avant la
nécessité absolue de prendre le temps de formaliser les notions abstraites nécessaires
à l'explicitation de la nature des objets que l'on souhaite définir.

Pour cela nous avons présenter la formalisation des séries entières effectuées au
sein du laboratoire et les conséquences pratiques :
\begin{itemize}
 \item définitions aisée de la fonction exponentielle et des fonctions
 trigonométriques usuelles,
 \item obtention de propriétés de ces fonctions comme corrollaires triviaux
 de résultats sur les séries entières (charactère $C^{\infty}$, lien entre les
 différentes fonctions, etc.),
 \item possibilité d'utiliser une tactique générique travaillant sur les séries
 entières lorsqu'on évoque ces fonctions
\end{itemize}

\subsection{Constructive axiomatic for the real numbers - Coq Workshop}

Le Coq Workshop nous a mis en contact avec les chercheurs les plus au courant du domaine. Une refondation des réels en Coq est un problème important et il était nécessaire à la fois de cerner un peu plus les approches existantes et de présenter la nôtre pour profiter des critiques des chercheurs du domaine. Notamment on a pu connaître les avancées récentes de C-CoRN, une approche des réels constructifs dont le principal inconvénient est sa lourdeur hiérarchique et une gestion moins fine de la distinction classique / constructive.

\subsection{Using reflection to solve some differential equations - Coq Workshop}

À l'inverse du public de THedu, le public du Coq Workshop est composé de chercheurs
habitués à utiliser Coq ce qui nous a permis de rentrer plus dans les détails de la
construction (élégante !) de la tactique par reflection.

Une potentielle amélioration de la robustesse du système de réification~\cite{adhoc}
du but courant nous a été suggérée à l'issue de la présentation.

\section{Devenir des membres de l'équipe}


L'intégralité des membres du laboratoire junior a décroché un M2
en informatique théorique et poursuit ses études sous la forme
d'une thèse en France et/ou à l'étranger.

\paragraph{Allais Guillaume} commence une thèse à l'Université de
Strathclyde (Glasgow, Écosse) intitulée « Concevoir la précision
à l'aide des types dépendants » sous la direction de Conor McBride.
Le but de ce travail est de rendre accessible aux programmeurs
l'expressivité des types dépendants afin de leur permettre de
développer plus aisément des programmes certifiés corrects.

\paragraph{Marthe Bonamy} commence une thèse au LIRMM (Montpellier)
sur les « procédures de déchargements pour résoudre des problèmes
d'optimisation dans les graphes » sous la direction de Christophe
Paul, Benjamin Lévêque et Alexandre Pinlou.
Le but de ce travail est d'approfondir l'étude de techniques de
preuves (notamment utilisées pour démontrer le théorème des 4 couleurs).

\paragraph{Sylvain Dailler} commence une thèse entre
l'Université d'Orléans et la Kochi University of Technology (Japon)
sur la « compilation certifiée de langages concurrents de haut niveau »
sous la direction de Frédéric Dabrowski et Frédéric Loulergue.
Le but de ce travail est de développer en théorie et en pratique des
outils certifiés corrects traduisant les langages concurrents utilisés
par les programmateurs en instructions machines.

\paragraph{Jean-Marie Madiot} commence une thèse en co-tutelle entre
le LIP et l'Université de Bologne sous la direction de Daniel Hirschkoff
et Davide Sangiorgi. Il s'intéressera à l'étude formelle des programmes
concurrents et plus particulièrement au développement de techniques et
outils de preuve pour la certification des programmes s'éxécutant en
parallèle.

\paragraph{Pierre-Marie Pédrot} commence une thèse au sein de l'équipe
INRIA PiR2 (Paris-Rocquencourt) sous la direction d'Alexis Saurin et Hugo
Herbelin. Son travail consistera à formaliser un  calcul avec ressources
équipé de types dépendants conciliant ainsi deux branches jusqu'alors
orthogonales de la logique: la théorie des types de Martin-Löf et la logique
linéaire de Jean-Yves Girard.



\bibliographystyle{amsplain}
\bibliography{biblio}

\end{document}

