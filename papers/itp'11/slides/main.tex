\documentclass{beamer}
\usetheme[titlepagelogo=coqtail_big,
          alternativetitlepage=true,% Use the fancy title page.
          ]{Torino}
\usepackage{stmaryrd}
\usepackage{fancyvrb}

\usecolortheme{coqtail}

\newcommand{\coqtail}{\textsc{coqtail}}
\newcommand{\B}{\mathbb{B}}
\newcommand{\E}{\mathcal{E}}
\newcommand{\N}{\mathbb{N}}
\newcommand{\R}{\mathbb{R}}

\author{Guillaume Allais}
\title{Using reflection to solve some differential equations}
\institute{\coqtail{} Junior Laboratory\\
	ENS Lyon}
\date{August, 26th}

\begin{document}

\begin{frame}[t,plain]
\titlepage
\end{frame}

\begin{frame}
\tableofcontents
\end{frame}

\section{Motivations}

\begin{frame}[fragile]{\coqtail{} defines new objects}
\begin{itemize}
  \item Power series

  \begin{verbatim}
      an  : Rseq
      rho : infinite_cv_radius an
      ===========================
      sum an rho : R -> R
  \end{verbatim}

  \item N$^{th}$ derivative

  \begin{verbatim}
      n   : nat
      f   : R -> R
      Dnf : D n f
      =========================
      nth_derive f Dnf : R -> R
  \end{verbatim}
\end{itemize}
\end{frame}

\begin{frame}[fragile]{With specific properties}
\begin{itemize}
  \item Trivial identities
  \begin{itemize}
    \item \verb!      sum an rho1 == sum an rho2!
    \item \verb!sum (an + bn) rab == sum an ra + sum bn rb!
  \end{itemize}
\bigskip
  \item Interactions
  \begin{itemize}
    \item A power series can be differentiated infinitely many times
    \item The shape of these derivatives is simple
  \end{itemize}
\end{itemize}

\bigskip
\uncover<2>{Do we really want to deal with this by hand?}
\end{frame}

\begin{frame}{Reflection}
\begin{itemize}
  \item A datatype representing formulas
  \item A semantics connecting the datatype to the formulas
\end{itemize}
\end{frame}

\section{A simple toy example}

\begin{frame}[fragile]{A simple toy example}
\begin{verbatim}
  Inductive side_equa : Set :=
    | y    : forall (p : nat) (k : nat), side_equa
    | plus : forall (s1 s2 : side_equa), side_equa.
\end{verbatim}
\end{frame}

\begin{frame}{First semantics}

\begin{itemize}
  \item From ASTs to power series

$$\begin{array}{lcrcl}
\llbracket & y (p, k) & \rrbracket_\R & \rho = & \left(\sum_n \rho(p) x^n\right)^{(k)}\\
\llbracket & plus(s_1,s_2) & \rrbracket_\R & \rho = & \llbracket s_1 \rrbracket_\R \rho + \llbracket s_2 \rrbracket_\R \rho
\end{array}$$
\end{itemize}
\end{frame}

\begin{frame}{Second semantics}
\begin{itemize}
  \item From ASTs to coefficients' sequences

$$\begin{array}{lcrcl}
\llbracket & y (p, k) & \rrbracket_\N & \rho = & \left(\frac{(n + k)!}{n!} \rho(p)_{n+k}\right)\\
\llbracket & plus(s_1,s_2) & \rrbracket_\N & \rho = & \llbracket s_1 \rrbracket_\N \rho + \llbracket s_2 \rrbracket_\N \rho
\end{array}$$
\end{itemize}
\end{frame}

\begin{frame}[fragile]{Main theorem}
We can talk about coefficients' sequences to prove equalities on the corresponding
power series.

\bigskip

$$\llbracket s_1 :=: s_2 \rrbracket_\N ({\tt map} ~ \pi_1 ~ \rho)$$
$$\Downarrow$$
$${\only<2>{\color{red}}\llbracket s_1 :=: s_2 \rrbracket_\R \rho}$$
\end{frame}

\begin{frame}[fragile]{Ltac}
\begin{itemize}
  \item Quoting
  \begin{itemize}
    \item<2-> $\verb!isconst s x! : \B$
    \item<3-> $\verb!add_var an rho env! : \N \star \E$
    \item<4-> $\verb!quote_side_equa env s x! : \E \star {\tt side\_equa}$
  \end{itemize}
\bigskip
  \item Normalizing
  \begin{itemize}
     \item<5-> $\verb!normalize_rec p s x! : {\tt unit}$
  \end{itemize}
\bigskip
  \item Solving
  \begin{itemize}
     \item<6> $\verb!solve_diff_equa! : {\tt unit}$
  \end{itemize}
\end{itemize}
\end{frame}

\begin{frame}[fragile]{Examples}

\begin{center}
\begin{minipage}{0.6\textwidth}
\begin{verbatim}
  an : Rseq
  ra : infinite_cv_radius an
  rb : infinite_cv_radius an
  ===========================
  sum an ra == sum an rb
\end{verbatim}
\end{minipage}
\end{center}

\bigskip

\begin{SaveVerbatim}{uncover2}
([(an, ra)] , (y(0,0), y(0,0)))
\end{SaveVerbatim}

\begin{SaveVerbatim}{uncover3}
  nth_derive (sum an ra) (D_infty_Rpser an ra 0) ==
  nth_derive (sum an ra) (D_infty_Rpser an ra 0)
\end{SaveVerbatim}

\begin{SaveVerbatim}{uncover4}
  an == an
\end{SaveVerbatim}

\uncover<2->{\UseVerbatim{uncover2}}

\uncover<3->{\UseVerbatim{uncover3}}

\uncover<4>{\UseVerbatim{uncover4}}
\end{frame}

\begin{frame}[fragile]
\begin{verbatim}
  an  : Rseq
  bn  : Rseq
  rab : infinite_cv_radius (an + bn + bn)
  ra  : infinite_cv_radius an
  rb  : infinite_cv_radius bn
  rc  : infinite_cv_radius bn
  =======================================
  sum (an + bn + bn) rab ==
  sum bn rb + sum an ra + sum bn rc
\end{verbatim}
\end{frame}

\section{All the features}

\begin{frame}
\begin{itemize}
  \item Just a toy example however...
  \begin{itemize}
    \item Fully automatic
    \item Quite reflects the actual implementation
  \end{itemize}
\end{itemize}
\end{frame}

\begin{frame}[fragile]{All the features}
\begin{verbatim}
  Inductive side_equa : Set :=
    | cst  : forall (r : R), side_equa
    | scal : forall (r : R) (s : side_equa), side_equa
    | y    : forall (p : nat) (k : nat) (a : R), side_equa
    | opp  : forall (s1 : side_equa), side_equa
    | min  : forall (s1 s2 : side_equa), side_equa
    | plus : forall (s1 s2 : side_equa), side_equa
    | mult : forall (s1 s2 : side_equa), side_equa.
\end{verbatim}
\end{frame}

\begin{frame}{Any questions?}

Sources: \url{http://coqtail.sf.net}

\end{frame}

\end{document}

