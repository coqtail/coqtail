\documentclass[submission,copyright]{eptcs}
\providecommand{\event}{THedu 2011} % Name of the event you are submitting to
\usepackage{breakurl}             % Not needed if you use pdflatex only.
\usepackage{amssymb, amsmath}

\newcommand{\coq}{Coq}
\newcommand{\coqtail}{\textsc{coqtail}}
\newcommand{\N}{\mathbb{N}}
\newcommand{\R}{\mathbb{R}}
\newcommand{\Rbar}{\overline{\mathbb{R}}}
\newcommand{\cvrw}{\texttt{Cv\_radius\_weak}}
\newcommand{\fcvr}{\texttt{finite\_cv\_radius}}
\newcommand{\icvr}{\texttt{infinite\_cv\_radius}}

\newtheorem{definition}{Definition}
\newtheorem{lemma}{Lemma}

\title{\coq{} with power series}
\author{XXX
\institute{Junior Laboratory \coqtail{}\\
Ens Lyon - France\\}
\email{XXX}}
\def\titlerunning{\coq{} with power series}
\def\authorrunning{XXX, \coqtail{} team}

\begin{document}
\maketitle

\begin{abstract}
Even if the trigonometric functions are defined in \coq{}'s standard
library, they were not described as power series because of a lack of
formalisation of this concept.
We present the strategy chosen to describe the power series over $\R$
and then advocates for the broad use of such abstract concepts by
showing the immediate benefits that one can get from this approach.

These benefits include the possibility to redefine the usual functions
($\sin$, $\cos$, $\exp$, ...) in a few lines, to get some of their
properties for free (eg. being in the $C^{\infty}$ class) and to prove
that they are solutions of particular differential equations just by
studying sequences over the reals.

\textbf{Files:} all the results mentioned in this paper are available on
\coqtail{}'s svn repository\footnote{see \url{http://sourceforge.net/projects/coqtail/develop}
and in particular \texttt{src/Reals/Rpser.v}} and will
be part of the next release.

\end{abstract}

\section{Formalization}

The main difference between what one can find in an undergraduate textbook
and our formalization of the power series is at the very begining of the
library. Our definition of the radius of convergence is much more informative
in an intuitionistic setting than the usual one.

This design choice allows us to get rid of the EM axiom in almost all
the proofs that usually use it because of the strenghtening of the
assumptions that it implies. This is however not harmful given that our
definition has been proved classicaly equivalent to the more traditional
one.

\subsection{The convergence radius}

The convergence radius $\rho$ of a power series whose general term is
$(a_n)_{n \in \N}$ is usually defined as a lowest upper bound (in
$\R \cup \left\lbrace +\infty \right\rbrace$):

 $$\rho(\sum_{n \in \N} a_n x^n) = sup \left\lbrace r \in \R ~|~
   \text{the sequence } \left|a_n r^n\right| \text{ is bounded}
   \right\rbrace$$

As being bounded is obviously not decidable, knowing that $r$ is the
convergence radius of $\sum_{n \in \N} a_n x^n$ is not sufficient to
show without using EM that for all x such that $x \in B(0,r)$, the
sequence $(\left| a_n x^n \right|)_{n \in \N}$ is bounded. That is why
we use an alternative definition which describes exactly the same
idea (being the lowest upper bound) but, by being more verbose, is
easier to use.

\begin{definition}[Rpser\_def] We say that $r$ is a weak convergence
radius if it is a lower bound for the convergence radius (ie $r$ belongs
to the set described previously).
$$\cvrw{}(a_n,r) = \left| a_n r^n \right| \text{ is bounded}$$
\end{definition}

From this definition, we can derive the definition of the finite
convergence radius\footnote{The extension to the infinite case is
straightforward and will not be described here. It is however formalized
and available on the repository.}.

\begin{definition}[Rpser\_def] The convergence radius is finite and
equal to $r$ if $r$ is both bigger than all the weak convergence radiuses 
and smaller or equal to all the reals that are to big to be weak
convergence radiuses.
$$\fcvr{}(a_n,r) =
\begin{array}{clcl}
        & \forall r', & 0 \le r' < r & \Rightarrow \cvrw{}(a_n,r') \\
 \wedge & \forall r', & r < r' & \Rightarrow \neg \cvrw{}(a_n,r')
\end{array}$$
\end{definition}

The classical equivalence between this definition and the usual one is proved
throught the two following lemmas. Unsurprisingly, the first implication does
not need the excluded middle axiom: our definition is stronger in an EM-free
setting than the usual one.

\begin{lemma}[Rpser\_base\_facts] $\fcvr{}(a_n,r) \Rightarrow r = sup \left\lbrace x |
\cvrw{}(a_n, x) \right\rbrace$ \end{lemma}

\begin{lemma}[Rpser\_base\_facts] $EM + r = sup \left\lbrace x |
\cvrw{}(a_n, x) \right\rbrace \Rightarrow \fcvr{}(a_n,r)$ \end{lemma}

Once that we have these definitions and that we know that they are
equivalent to their usual counterparts, we can start manipulating them.
The first important tool that we can get is Alembert's ratio criterion.
It is rather useful to prove that a particular power series has a given
convergence radius (this result is for example used to prove that $\exp$
has an infinite convergence radius which implies automatically that this
is also the case for $\cos$ and $\sin$).

\begin{lemma}[Rpser\_cv\_facts] A weak version of Alembert's ratio
criterion given that for all $n$, $a_n \neq 0$:

$$\lim\limits_{n \to + \infty}\frac{a_{n+1}}{a_n} = \lambda \neq 0
  \Rightarrow \forall \left| r \right| < \frac{1}{\left| \lambda \right|
  }, \cvrw{}(a_n,r)$$
\end{lemma}

\subsection{Sum of a power series}

When we know what is the power series' convergence radius, we can start
summing it on the appropriate domain. Our beloved tool to define the sum
of a power series is obviously Abel's lemma which states that given a
convergence radius, we can sum the power series inside the corresponding
ball.

\begin{lemma}[Rpser\_radius\_facts] $$\cvrw{}(a_n,r) \Rightarrow
\forall x \in B(0,r), \exists l, \sum_{n=0}^{+\infty} a_n x^n = l$$
\end{lemma}

\begin{definition}[Rpser\_sums] From this lemma we can construct the functions
(namely \texttt{weaksum\_r}, \texttt{sum\_r} and \texttt{sum}) that given either
$\cvrw{}(a_n,r)$, $\fcvr{}(a_n,r)$ or $\icvr{}(a_n)$ output the piecewise-defined
function: $$x \mapsto \left\lbrace
\begin{array}{ll}
\sum_{n=0}^{+\infty} a_n x^n & \text{ if } x \text{ is inside the convergence
disk}\\
0 & \text{otherwise}
\end{array}\right.$$
\end{definition}

\subsection{Derivative of a power series}

%TODO: add the description of the derivative of the power series

As the derivative of a power series is a power series, it is therefore trivial
to show that the function defined as the sum belongs to the $C^{\infty} class$.
As a consequence, we can get an explicit description of the $n^{th}$ derivative
of the sum by a simple induction:

\begin{lemma}[\texttt{Rpser\_derivative\_facts}] Explicit description of the
$k^{th}$ derivative of the power series defined by $(a_n)_{n\in \N}$:
$$(\sum_{n=0}^{+\infty} a_n x^n)^{(k)} = \sum_{n=0}^{+\infty}
\frac{(n + k)!}{n!} a_{n+k} x^n$$
\end{lemma}

\section{Uses}

\subsection{Defining usual functions}

Thanks to all the formalization work, we can now define all the usual
functions in a few lines. We begin by defining the exponential using
d'Alembert's ratio test to prove that its convergence radius is infinite
(which is rather easy).

Then the compatibility of \cvrw{} with the pointwise order on the
sequences:

$$\left(\left|b_n\right| \le_{pointwise} \left|a_n\right|\right)
  \Rightarrow \forall r, \cvrw{}(a_n,r) \Rightarrow \cvrw{}(b_n,r)$$

gives us for free the fact that cosine and sine also have an infinite
convergence radius.

%TODO: add description of exp, cos, sin

\subsection{Comparison to the standard library}

We have therefore been able to define $\exp$, $\sin$ and $\cos$ in less
than 80 lines of Coq source code. It has to be compared to the hundreds
of lines and the ad-hoc arguments (convergence of alternating series)
that the standard library dedicates to the definition of these exact
same functions.

Not only defining these functions is much more easier, but we get for
free their derivability (and even their being in the $C^{\infty}$ class),
the exact shape of their $n^{th}$ derivative and therefore we can easily
prove the relation that exists between them.

\subsection{Finding solutions of linear differential equations}

%TODO: cf. Dequa_examples

\section{Extensions}

%TODO: Cpser & add multiplication to the differential equations

\end{document}
