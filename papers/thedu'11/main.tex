\documentclass[submission,copyright]{eptcs}
\providecommand{\event}{THedu 2011} % Name of the event you are submitting to
\usepackage{breakurl}             % Not needed if you use pdflatex only.
\usepackage{amssymb, amsmath}

\newcommand{\coq}{Coq}
\newcommand{\coqtail}{\textsc{coqtail}}
\newcommand{\R}{$\mathbb{R}$}

\title{\coq{} with power series}
\author{XXX
\institute{Junior Laboratory \coqtail{}\\
Ens Lyon - France\\}
\email{XXX}}
\def\titlerunning{\coq{} with power series}
\def\authorrunning{XXX, \coqtail{} team}

\begin{document}
\maketitle

\begin{abstract}
This paper is based on a \coq{} formalisation of the power series over \R{}. It
presents briefly the strategy adopted to talk about these objects and then
advocates for the use of such abstract concepts by showing the benefits that
one gets for free.

These benefits includes the possibility to define the usual functions ($\sin$,
$\cos$, $\exp$, ...) in a few lines, to get some of their properties for free
(eg. being in the $C^{\infty}$ class) and to find solutions of some linear
differential equations just by looking at sequences over \R{}.

\end{abstract}

\section{Formalisation}

\section{Uses}
\subsection{Defining usual functions}
\subsubsection{Comparison to the standard library}

\subsection{Finding solutions of linear differential equations}

\section{Extensions}


\end{document}
