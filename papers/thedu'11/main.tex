\documentclass[submission,copyright]{eptcs}
\providecommand{\event}{THedu 2011} % Name of the event you are submitting to
\usepackage{breakurl}             % Not needed if you use pdflatex only.
\usepackage{amssymb, amsmath}

\newcommand{\coq}{Coq}
\newcommand{\coqtail}{\textsc{coqtail}}
\newcommand{\N}{\mathbb{N}}
\newcommand{\R}{\mathbb{R}}
\newcommand{\Rbar}{\overline{\mathbb{R}}}

\DeclareMathOperator{\C}{\mathtt{An\_cst}}
\DeclareMathOperator{\D}{\mathtt{An\_deriv}}
\DeclareMathOperator{\Dn}{\mathtt{An\_nth\_deriv}}
\DeclareMathOperator{\IN}{\mathtt{interp\_N}}
\DeclareMathOperator{\cvrw}{\mathtt{Cv\_radius\_weak}}
\DeclareMathOperator{\fcvr}{\mathtt{finite\_cv\_radius}}
\DeclareMathOperator{\icvr}{\mathtt{infinite\_cv\_radius}}

\newtheorem{definition}{Definition}
\newtheorem{lemma}{Lemma}
\newtheorem{theorem}{Theorem}

\title{\coq{} with power series}
\author{Guillaume Allais
\institute{Junior Laboratory \coqtail{}\\
Ens Lyon - France\\}
\email{guillaume.allais@ens-lyon.org}}
\def\titlerunning{\coq{} with power series}
\def\authorrunning{\coqtail{}}

\begin{document}
\maketitle

\begin{abstract}
In the interactive theorem prover \coq{}\cite{coq}, trigonometric
functions are defined in the standard library. However there are not
directly described as power series  due to a lack of formalization of
this concept. We present a strategy to describe the power series over
$\R$ and then advocate for the broad use of abstract concepts in the
standard library by showing the immediate benefits that one can get.

Thanks to the power series framework we are able to redefine the usual
functions (e.g. $\sin$, $\cos$, $\exp$) in less than 80 lines,
to get some of their properties for free (e.g. being in the $C^{\infty}$
class) and to prove that they are solutions of particular differential
equations just by studying sequences over the reals.

\textbf{Files:} All the implementation mentioned in this paper are
available for download via \coqtail{}'s svn repository\footnote{see
\url{http://sourceforge.net/projects/coqtail/develop}
and in particular \texttt{src/Reals/Rpser.v}.}.

\end{abstract}

\section{Formalization}

Power series are a rather intensional mathematical notion; therefore, its 
formalization in an interactive theorem prover is really close to  
textbooks' presentation. Except for one point: the definition of the
convergence radius.

In an intuitionistic setting, our definition of the radius of convergence
is much more informative than the standard one.  This is however not
harmful given that our definition has been proved classically equivalent
to the standard one. This design choice allows us to get rid of the
excluded-middle (henceforth EM) axiom in most proofs that usually use
it. As a drawback, one could expect the lemmas whose conclusion is the
existence of a radius of convergence not to be provable anymore. It is
fortunately not the case\footnote{At least for d'Alembert's ratio criterion
which is one of the key lemmas that have this shape.}.


\subsection{The convergence radius}

The convergence radius $\rho$ of a power series whose coefficients is
$(a_n)_{n \in \N}$ is usually defined as a lowest upper bound (in
$\R \cup \left\lbrace +\infty \right\rbrace$):

 $$\rho(\sum_{n \in \N} a_n x^n) = \sup \left\lbrace r \in \R ~|~
   \text{the sequence } \left|a_n r^n\right| \text{ is bounded}
   \right\rbrace$$

As being bounded is obviously not decidable, knowing that $r$ is the
convergence radius of $\sum_{n \in \N} a_n x^n$ is not sufficient to
show without using EM that the sequence $(\left| a_n x^n \right|)
_{n \in \N}$ is bounded for all $x$ in $B_r(0)$. That is why
we use a more verbose definition which describes exactly the same
idea (being the lowest upper bound) and is easier to use.

\begin{definition}[Rpser\_def] We say that $r$ is a weak convergence
radius if it is a lower bound for the convergence radius (ie. $r$ belongs
to the disk of convergence).
$$\cvrw{}(a_n,r) = \left| a_n r^n \right| \text{ is bounded}$$
\end{definition}

From this definition, we can obtain the definition of the finite
convergence radius\footnote{The extension to the infinite case is
straightforward: the convergence radius is infinite if and only if
all the reals are weak convergence radiuses.}.

\begin{definition}[Rpser\_def] The convergence radius is finite and
equal to $r$ if $r$ is both bigger than all the weak convergence radiuses 
and smaller or equal to all the reals that are too big to be weak
convergence radiuses.
$$\fcvr{}(a_n,r) =
\begin{array}{clcl}
        & \forall r', & 0 \le r' < r & \Rightarrow \cvrw{}(a_n,r') \\
 \wedge & \forall r', & r < r' & \Rightarrow \neg \cvrw{}(a_n,r')
\end{array}$$
\end{definition}

The classical equivalence between this definition and the usual one is proved
throught the two following lemmas. Unsurprisingly, the first implication does
not need the excluded middle axiom: our definition is stronger in an EM-free
setting than the usual one.

\begin{lemma}[Rpser\_base\_facts] $\fcvr{}(a_n,r) \Rightarrow r = \sup
\left\lbrace x | \cvrw{}(a_n, x) \right\rbrace$ \end{lemma}

\begin{lemma}[Rpser\_base\_facts] $EM + r = \sup \left\lbrace x |
\cvrw{}(a_n, x) \right\rbrace \Rightarrow \fcvr{}(a_n,r)$ \end{lemma}

The first important tool that we can get is d'Alembert's ratio criterion.
It can be used to prove that a particular power series has a given
convergence radius; this result is for example used to prove that $\exp$
has an infinite convergence radius from which we can easily deduce
that both $\cos$ and $\sin$ also do.

\begin{lemma}[Rpser\_cv\_facts] A weak version of d'Alembert's ratio
criterion given that for all $n$, $a_n \neq 0$:

$$\lim\limits_{n \to + \infty}\frac{a_{n+1}}{a_n} = \lambda \neq 0
  \Rightarrow \forall \left| r \right| < \frac{1}{\left| \lambda \right|
  }, \cvrw{}(a_n,r)$$
\end{lemma}

\subsection{Sum of a power series}

When we know what is the power series' convergence radius, we can start
summing it on the appropriate domain. Our beloved tool to define the sum
of a power series is obviously Abel's lemma which states that given a
convergence radius, we can sum the power series inside the corresponding
ball.

\begin{lemma}[Rpser\_radius\_facts] $$\cvrw{}(a_n,r) \Rightarrow
\forall x \in B_r(0), \exists l, \sum_{n=0}^{+\infty} a_n x^n = l$$
\end{lemma}

\begin{definition}[Rpser\_sums] From this lemma we can construct the functions
(namely \texttt{weaksum\_r}, \texttt{sum\_r} and \texttt{sum}) that given either
$\cvrw{}(a_n,r)$, $\fcvr{}(a_n,r)$ or $\icvr{}(a_n)$ output the piecewise-defined
function: $$x \mapsto \left\lbrace
\begin{array}{ll}
\sum_{n=0}^{+\infty} a_n x^n & \text{ if } x \text{ is inside the convergence
disk}\\
0 & \text{otherwise}
\end{array}\right.$$
\end{definition}

\subsection{Derivative of a power series}

The formalization of the derivative of a power series is done in two steps.
First of all, we define the formal derivative of a power series and prove
that this definition makes sense (e.g. that the sum exists):

\begin{definition}[Rpser\_derivative] The formal derivative of the power
series defined by $(a_n)$ is the one defined by:
$$\D{}(a_n) = (n + 1) * a_{n+1}$$
\end{definition}

\begin{lemma}[Rpser\_radius\_facts] And its convergence radius is exactly
the same as the one of the original series: $$\fcvr{}(a_n,r) \Leftrightarrow
\fcvr{}(\D{}(a_n),r)$$\end{lemma}

We now know that the formal derivative is summable but we still have to
somehow explicit the relation that exists between these two power series.
This is maybe the most complex part of the library; it uses a theorem on
sequences of functions (see infra) plus the fact that the sequence of
the partial sums of a power series converges normally (thus uniformly)
inside the disk of convergence.

\begin{theorem}[RFsequence\_facts] On the ball $B_r(c)$ if
$(f_n)_{n \in \N}$ is a sequence of derivable functions and if the
following limits exist: $$ \begin{array}{ccc}
f_n \xrightarrow{n \to +\infty} f & \text{and} &
f_n' \xrightarrow[CVU]{n \to +\infty} g\end{array}$$ then $f$ is
derivable and its derivative is $g$.\end{theorem}

As a consequence, we can conclude that the power series are derivable
and that their derivative is precisely the sum of the formal derivative.
As the derivative of a power series is another power series, it is
trivial to show that the function defined as the sum belongs to the
$C^{\infty}$ class\footnote{See the C\_n\_* files for results on
function classes.}. A simple induction can even give us the explicit
description of the $n^{th}$ derivative of a sum:

\begin{definition}[Rpser\_def] The formal $k^{th}$ derivative of the power
series defined by $(a_n)$ is the one defined by:
$$\Dn{}(a_n,k)  = \frac{(n + k)!}{n!} a_{n+k}$$
\end{definition}

\begin{lemma}[Rpser\_derivative\_facts] Explicit description of
the $k^{th}$ derivative of the power series defined by $(a_n)_{n\in \N}$:
$$(\sum_{n=0}^{+\infty} a_n x^n)^{(k)} = \sum_{n=0}^{+\infty}
\Dn{}(a_n,k) x^n$$\end{lemma}

\section{Applications}

\subsection{Defining usual functions}

Thanks to all the formalization work, we can now define all the usual
functions in a few lines. We begin by defining the exponential using
d'Alembert's ratio test to prove that its convergence radius is infinite
(which is rather easy).

Then we can get for free the fact that cosine and sine also have an
infinite convergence radius by using one of the basic lemmas:

\begin{lemma}[Rpser\_base\_facts] $\cvrw{}$ is compatible with the
pointwise order on the sequences over reals:

$$\left(\left|b_n\right| \le_{pointwise} \left|a_n\right|\right)
  \Rightarrow \forall r, \cvrw{}(a_n,r) \Rightarrow \cvrw{}(b_n,r)$$
\end{lemma}

We have been able to define $\exp$, $\sin$ and $\cos$ in less than 80
lines of Coq source code. It has to be compared to the hundreds
of lines and the ad-hoc arguments (e.g. convergence of alternating series)
that Coq's standard library dedicates to the definition of these exact
same functions.

Not only defining these functions is much more easier, but we get for
free their derivability (and even their being in the $C^{\infty}$ class),
the exact shape of their $n^{th}$ derivative and therefore we can easily
prove the relation that exists between them.

\subsection{Finding solutions of linear differential equations}

On top of the power series formalization, we constructed a small library
that describes linear differential equations of arbitrary shape by
reflection. It is based on two components:

\begin{itemize}
 \item A datatype \texttt{side\_equa} that is a descriptor of linear 
   differential equations. A differential equation is a pair $(E_1,E_2)$
	of side equations (usually written $E_1 :=: E_2$).
 
 \item A (set of) semantics that given an inhabitant of this datatype
	and an environment outputs a proposition.
\end{itemize}

\begin{definition}[Dequa\_def] The grammar of the \texttt{side\_equa}
datatype is (where $c$ is a real constant):
$$E ::= c ~|~ y_i^{(k)} ~|~ - E ~|~ E + E$$\end{definition}


\begin{definition}[Dequa\_def] The semantics that translates differential
equations into propositions on sequences over $\R$ is define in two steps.
First of all we define recursively the interpretation of the side
of an equation:$$\begin{array}{lclcl}
\IN{}& & & : & \texttt{side\_equa} \rightarrow Rseq~list \rightarrow Rseq \\
\IN{}(& c &, \rho) & = & \C{}(c)\\
\IN{}(& y_i^{(k)} &, \rho) & = & \Dn{}(\rho(i),k) \\
\IN{}(& - E &, \rho) & = & - \IN{}(E,\rho)\\
\IN{}(& E_1 + E_2 &, \rho) & = & \IN{}(E_1,\rho) + \IN{}(E_2,\rho)\\
\end{array}$$
and then we can construct the interpretation function:
$$\left[\left| E_1 :=: E_2 \right|\right]_\N \rho = \forall n,
\IN{}(E_1,rho) = \IN{}(E_2,\rho)$$\end{definition}

We can define similarly the semantics $\left[\left| \_ \right|\right]_\R$
that translates differential equations into propositions on power series
which allows us to state general theorems about linear differential
equations.

\begin{lemma}[Dequa\_facts] Our main result states that given a context
$\rho$ of sequences of coefficients, we have (provided that the involved
power series have an infinite radius of convergence):
$$\left[\left| E_1 :=: E_2 \right|\right]_\N \rho \Rightarrow
\left[\left| E_1 :=: E_2 \right|\right]_\R \rho$$

ie. we can prove that some functions (sums of power series) are solutions
of a given differential equation by proving results on their coefficients.
\end{lemma}

By using this theorem, one can prove in less than 20 lines that the
exponential is a solution of the equation $y^{(n+1)} = y^{(n)}$. Without
much more work, one can also prove that cosine and sine are solutions
of $y^{(2)} = - y$.

\section{Extensions}

\subsection{Power series on other carriers}

\coqtail{} already has a basic library on the power series over the complex
numbers\footnote{See src/Complex/Cpser\_*.v} (definitions, differentiability,
derivability). A future work could be to adapt the work done on the real
numbers to the complex numbers. A better idea might be to formalize the
power series on a more general structure (e.g. a ring equipped with a norm).

\subsection{Improving the library on differential equations}

Even if the work on the differential equations is already quite
convenient, it is still rather limited: one can only derive variables
and not composed expressions, there is currently no built-in minus
function or multiplication by a constant and it is impossible to talk
about the product of power series.

Describing the explicit shape of the product of two power series (the
power series which is defined with the cauchy product of the sequences
defining the two others) is a possible future work.

\nocite{*}
\bibliographystyle{eptcs}
\bibliography{biblio}

\section{Acknowledgements}

Special thanks to Sylvain and Marc for their useful comments and
suggestions.

\end{document}
