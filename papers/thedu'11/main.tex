\documentclass[submission,copyright]{eptcs}
\providecommand{\event}{THedu 2011} % Name of the event you are submitting to
\usepackage{breakurl}             % Not needed if you use pdflatex only.
\usepackage{amssymb, amsmath}

\newcommand{\coq}{Coq}
\newcommand{\coqtail}{\textsc{coqtail}}
\newcommand{\N}{\mathbb{N}}
\newcommand{\R}{\mathbb{R}}
\newcommand{\Rbar}{\overline{\mathbb{R}}}
\newcommand{\cvrw}{\texttt{Cv\_radius\_weak}}
\newcommand{\fcvr}{\texttt{finite\_cv\_radius}}
\newcommand{\icvr}{\texttt{infinite\_cv\_radius}}

\newtheorem{definition}{Definition}
\newtheorem{lemma}{Lemma}

\title{\coq{} with power series}
\author{XXX
\institute{Junior Laboratory \coqtail{}\\
Ens Lyon - France\\}
\email{XXX}}
\def\titlerunning{\coq{} with power series}
\def\authorrunning{XXX, \coqtail{} team}

\begin{document}
\maketitle

\begin{abstract}
This paper is based on a \coq{} formalization of the power series over $\R$. It
presents briefly the strategy chosen to deal with these objects in \coq{} and
then advocates for the use of such abstract concepts by showing the benefits
that one gets for free.

These benefits includes the possibility to define the usual functions ($\sin$,
$\cos$, $\exp$, ...) in a few lines, to get some of their properties for free
(eg. being in the $C^{\infty}$ class) and to find solutions of some linear
differential equations just by looking at sequences over $\R$.

\textbf{Files:} all the results mentioned in this paper are available on
\coqtail{}'s svn repository\footnote{see \url{http://sourceforge.net/projects/coqtail/develop}
and in particular \texttt{src/Reals/Rpser.v}} and will
be part of the next release.

\end{abstract}

\section{Formalization}

Our formalization of the power series follows more or less what one could find
in an undergraduate textbook. There is however one important difference on the
definition of the convergence radius: our definition is much more informative
than the usual one.

This design choice allows us to get rid off the EM axiom in almost all the
proofs that usually use it without being harmful: it is proved to be classically
equivalent to the traditional definition.

\subsection{The convergence radius}

The convergence radius of a power series $\sum_{n \in \N} a_n x^n$ is usually
defined as the lowest upper bound (in $\R \cup \left\lbrace +\infty \right
\rbrace$) of the set $\left\lbrace r \in \R ~|~ \text{the sequence } \left|
a_n r^n\right| \text{ is bounded} \right\rbrace$.
As being bounded is obviously not decidable, knowing that $r$ is the convergence
radius of $\sum_n a_n x^n$ (written $\rho (\sum_n a_n x^n) = r$) is not
sufficient to show that $\forall x \in B(0,r), \left| a_n x^n \right| \text{ is
bounded}$ without using EM.

That is why we use an alternative definition which describes exactly the same
concept but, by being more verbose, is easier to use.

\begin{definition}[Rpser\_def] \cvrw{}$(a_n,r)$ means that the convergence
radius of $\sum_{n \in \N} a_n x^n$ is greater or equal to $r$:
$$\cvrw{}(a_n,r) = \left| a_n r^n \right| \text{ is bounded}$$
\end{definition}

\begin{definition}[Rpser\_def]\fcvr{}$(a_n,r)$ means that the
convergence radius of $\sum_{n \in \N} a_n x^n$ is finite and equal to $r$ ie.
that for all $x$ smaller than $r$, $\left|a_n x^n\right|$ is bounded and
that for all $x$ bigger than $r$,  $\left|a_n x^n\right|$ is not.
$$\fcvr{}(a_n,r) =
\begin{array}{clcl}
        & \forall r', & 0 \le r' < r & \Rightarrow \cvrw{}(a_n,r') \\
 \wedge & \forall r', & r < r' & \Rightarrow \neg \cvrw{}(a_n,r')
\end{array}$$
\end{definition}

The classical equivalence between this definition and the usual one is proved
throught the two following lemmas. Unsurprisingly, the first implication does
not need the excluded middle axiom: our definition gives more information than
the usual one.

\begin{lemma}[Rpser\_base\_facts] $\fcvr{}(a_n,r) \Rightarrow r = sup \left\lbrace x |
\cvrw{}(a_n, x) \right\rbrace$ \end{lemma}

\begin{lemma}[Rpser\_base\_facts] $EM + r = sup \left\lbrace x |
\cvrw{}(a_n, x) \right\rbrace \Rightarrow \fcvr{}(a_n,r)$ \end{lemma}

\begin{definition}[Rpser\_def] When the convergence radius is infinite we use
the unary predicate \icvr{}: $$\icvr{}(a_n) = \forall r, \cvrw{}(a_n,r)$$
\end{definition}

\subsection{Sum of a power series}

The main tool to define the sum of a power series is Abel's lemma which states
that given a convergence radius, we can sum the power series inside the
corresponding ball.

\begin{lemma}[Rpser\_radius\_facts] $$\cvrw{}(a_n,r) \Rightarrow
\forall x \in B(0,r), \exists l, \sum_{n=0}^{+\infty} a_n x^n = l$$
\end{lemma}

\begin{definition}[Rpser\_sums] From this lemma we can construct the functions
(namely \texttt{weaksum\_r}, \texttt{sum\_r} and \texttt{sum}) that given either
$\cvrw{}(a_n,r)$, $\fcvr{}(a_n,r)$ or $\icvr{}(a_n)$ output the piecewise-defined
function: $$x \mapsto \left\lbrace
\begin{array}{ll}
\sum_{n=0}^{+\infty} a_n x^n & \text{ if } x \text{ is inside the convergence
disk}\\
0 & \text{otherwise}
\end{array}\right.$$
\end{definition}

\subsection{Derivative of a power series}

%TODO: add the description of the derivative of the power series

As the derivative of a power series is a power series, it is therefore trivial
to show that the function defined as the sum belongs to the $C^{\infty} class$.
As a consequence, we can get an explicit description of the $n^{th}$ derivative
of the sum by a simple induction:

\begin{lemma}[\texttt{Rpser\_derivative\_facts}] Explicit description of the
$k^{th}$ derivative of the power series defined by $(a_n)_{n\in \N}$:
$$(\sum_{n=0}^{+\infty} a_n x^n)^{(k)} = \sum_{n=0}^{+\infty}
\frac{(n + k)!}{n!} a_{n+k} x^n$$
\end{lemma}

\section{Uses}

\subsection{Defining usual functions}

Thanks to all the formalization work, we can now define all the usual functions
in a few lines using d'Alembert's ratio test and the compatibility of \cvrw{}
with the pointwise order on the sequences in order to prove that the considered
power series have an infinite radius of convergence.

%TODO: add description of exp, cos, sin

\subsection{Comparison to the standard library}

%TODO: recall the definitions of the std lib's exp, cos, sin
%TODO: compare the length of the definitions

\subsection{Finding solutions of linear differential equations}

%TODO: cf. Dequa_examples

\section{Extensions}

%TODO: Cpser & add multiplication to the differential equations

\end{document}
