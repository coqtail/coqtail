\documentclass{beamer}
\usepackage[utf8x]{inputenc}
\usepackage{fancyvrb}
\usepackage{amsmath, amssymb}
\usetheme[titlepagelogo=coqtail_big,
          alternativetitlepage=true,% Use the fancy title page.
          ]{Torino}
\usecolortheme{coqtail}

\newcommand{\coqtail}{\textsc{coqtail}}
\newcommand{\rsequence}{\texttt{Rsequence}}
\newcommand{\rpser}{\texttt{Rpser}}
\newcommand{\ring}{\texttt{ring}}
\newcommand{\field}{\texttt{field}}
\newcommand{\solve}{\texttt{solve\_diff\_equa}}

\DeclareMathOperator{\Dn}{\mathtt{An\_nth\_deriv}}
\DeclareMathOperator{\cvrw}{\mathtt{Cv\_radius\_weak}}
\DeclareMathOperator{\fcvr}{\mathtt{finite\_cv\_radius}}
\DeclareMathOperator{\icvr}{\mathtt{infinite\_cv\_radius}}

\DeclareMathOperator{\T}{\mathcal{T}}

\newcommand{\N}{\mathbb{N}}
\newcommand{\R}{\mathbb{R}}

\DeclareMathOperator{\Interp}{\mathtt{interp}}
\DeclareMathOperator{\Sem}{\left[\left| E_1 :=: E_2 \right|\right]}
\DeclareMathOperator{\SemR}{\Sem_{\R} \rho}
\DeclareMathOperator{\SemN}{\Sem_{\N} \rho}
\DeclareMathOperator{\IR}{\Interp_{\R}}
\DeclareMathOperator{\IN}{\Interp_{\N}}


\author{Guillaume Allais}
\title{Coq with power series}
\institute{\coqtail{} Junior Laboratory\\
	ENS Lyon}
\date{July, 31st}

\begin{document}

\begin{frame}[t,plain]
\titlepage
\end{frame}

\begin{frame}{Why? - \coqtail{}}
\begin{itemize}
 \item We wanted to:
   \begin{itemize}
     \item Tackle undergraduate programs
     \item Prove nice results
     \item Produce clean and reusable libraries
   \end{itemize}
   \bigskip
 \item We needed:
   \begin{itemize}
     \item Good libraries \uncover<2->{$\Rightarrow$ \rsequence{} (Pédrot)}
     \item Good tactics
   \end{itemize}
\end{itemize}
\end{frame}

\begin{frame}[fragile,t]{Why? - \rpser{}}
\begin{verbatim}
  Definition cos_in (x l:R) : Prop :=
    infinite_sum (fun i:nat => cos_n i * x ^ i) l.
\end{verbatim}

\begin{SaveVerbatim}{ecos}
  Lemma exist_cos : forall x:R, { l:R | cos_in x l }.
\end{SaveVerbatim}

\begin{SaveVerbatim}{cos}
  Definition cos (x:R) : R := let (a,_) :=
    exist_cos (Rsqr x) in a.
\end{SaveVerbatim}

\uncover<2->{\UseVerbatim{ecos}}
\uncover<3->{\UseVerbatim{cos}}
\uncover<4->
{
\bigskip
\color{red}{But $\cos$ is much more than just a series!}
}
\end{frame}

\begin{frame}{Defining power series}
\begin{itemize}
\item Convergence
  \begin{itemize}
    \item Convergence disc
    \item Criterion
  \end{itemize}
\item Sums
  \begin{itemize}
    \item Abel's lemma
    \item Compatibility with common operations
    \item Formal derivatives
  \end{itemize}
\end{itemize}
\end{frame}

\begin{frame}{Convergence radius}
\begin{itemize}
  \item The usual definition
    $$\rho\left(\sum_{n \in \N} a_n x^n\right) = \only<1>{\sup}
    \only<2->{{\color{red} \sup }} \left\lbrace
     r \in \R ~|~\text{the sequence } \left|a_n r^n\right|
     \text{ is bounded}\right\rbrace$$
  \item<2-> But being a lub is not really informative!
  \begin{itemize}
    \item<3-> The convergence disk is convex
    \begin{itemize}
      \item<4-> But being bounded is not decidable
      \only<4>{$$o_{i,j} (n)= \left\lbrace
        \begin{array}{ll}
          0 & \text{if } \T_i(j) \text{ stops in less than } n \text{ steps}\\
          n & \text{otherwise}
         \end{array}\right.$$}
      \item<5-> Hence not provable without EM
    \end{itemize}
  \end{itemize}
\end{itemize}
\end{frame}

\begin{frame}{Our definition}
\begin{itemize}
  \item Being inside the convergence radius:
  $$\cvrw{}(a_n,r) = \left| a_n r^n \right| \text{ is bounded}$$
  \item Having a finite radius of convergence:
  $$\fcvr{}(a_n,r) =$$
  $$\begin{array}{clcl}
    & \forall r', & 0 \le r' < r & \Rightarrow \cvrw{}(a_n,r') \\
    \wedge & \forall r', & r < r' & \Rightarrow \neg \cvrw{}(a_n,r')
    \end{array}$$
\end{itemize}
\end{frame}

\begin{frame}{Applications}
\begin{itemize}
 \item Usual functions defined in a couple of lines.
  \begin{itemize}
     \item $\exp$
     \item $\cos$, $\sin$
  \end{itemize}
 \item Properties for free
  \begin{itemize}
    \item derivability
    \item shape of the n$^{th}$ derivative
  \end{itemize}
\end{itemize}
\end{frame}

\begin{frame}{Build tactics on top of this}
\begin{itemize}
  \item What is annoying when proving lemmas?
  \begin{itemize}
    \item Proving obvious equalities
    \item Compatibility with common operations
    \item Formal derivatives
  \end{itemize}
  \item How to avoid proving everything by hand?
  \begin{itemize}
    \item \ring{}, \field{}
    \item \solve{}
  \end{itemize}
\end{itemize}
\end{frame}

\begin{frame}{Why using reflection?}
\begin{itemize}
 \item Add more guarantees to your tactic
 \item Avoid the manipulation of huge terms
 \item Replace proofs by computations
 \item Easy to extend
\end{itemize}
\end{frame}

\begin{frame}{Simple remarks}
\begin{itemize}
  \item Sums of power series are extentional:
    $$a_n \equiv b_n \Rightarrow
   \sum_n a_n x^ n \equiv \sum_n b_n x ^ n$$
  \item Sums of power series are compatible with addition:
    $$\sum_n (a_n + b_n) x^ n \equiv \sum_n a_n x ^ n + \sum_n b_n x ^ n$$
  \item We know the exact shape of the n$^th$ derivative:
    $$(\sum_n a_n x^ n)^{(k)} \equiv \sum_n \Dn a_n x ^ n$$
\end{itemize}
\end{frame}

\begin{frame}{\solve{} - A very basic version}
\begin{itemize}
 \item Side equations: $E ::= y_i^{(k)} ~|~ E + E$
 \item<2-> Equations: $E1 :=: E2$
 \item<3-> Two semantics: talking about power series or sequences over $\R$
\end{itemize}
\end{frame}

\begin{frame}{$\SemR = ?$}
\begin{itemize}
\item $\IR$ is the trivial semantics à la Tarski that one could expect:
\only<1>{
$$\IR (y^{k}_i, \rho) = \left(\sum_n \rho(i)_n x ^ n\right)^{(k)}$$
$$\IR (E_1 + E_2, \rho) = \IR(E_1, \rho) + \IR(E_2, \rho)$$}
\item<2-> $\SemR$ is just the use $\IR$ on both sides of the equation:

$$\SemR = (\IR(E_1, \rho) \equiv \IR(E_2, \rho))$$
\end{itemize}
\end{frame}

\begin{frame}{Thanks for your attention!}

\begin{block}{More information available online:}
\url{http://coqtail.sf.net}
\end{block}

\end{frame}

\end{document}

