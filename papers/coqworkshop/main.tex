\documentclass[a4paper,11pt]{article}
\usepackage{amsmath, amssymb, amsthm, verbatim, fullpage, hyperref}
\usepackage[utf8x]{inputenc}


\newcommand{\R}{\mathbb{R}}
\newcommand{\Q}{\mathbb{Q}}
\newcommand{\Z}{\mathbb{Z}}
\newcommand{\N}{\mathbb{N}}
\newcommand{\Type}{\mbox{Type}}
\newcommand{\Prop}{\mbox{Prop}}
\newcommand{\coqtail}{\textsc{coqtail}}

\newtheorem{lemma}{Lemma}
\newtheorem{definition}{Definition}
\newtheorem{example}{Example}
\newtheorem{remark}{Remark}

\theoremstyle{definition}
\theoremstyle{remark}

\parskip=3pt

\begin{document}

\title{Constructive axiomatic for the real numbers}
\author{Jean-Marie Madiot \\ Pierre-Marie Pédrot \\ Junior Laboratory \coqtail{}\\ Ens Lyon - France\\}
\date{May 23, 2011}
%\email{jean-marie.madiot@ens-lyon.org}}

\maketitle 

  Reasoning about real numbers in Coq can be very tedious. The standard library of Coq is classical and using it means forgetting about constructive proofs. C-CoRN is constructive in every way and forbids classical statements. Moreover its hierarchical structure is not lightweight enough. We propose a short axiomatic that can handle both constructive and classical mathematical statements.

  The implementation in Coq is available at \coqtail's repository\footnote{\url{https://sourceforge.net/projects/coqtail/develop} in {\tt src/Fresh/Reals}}

\section{Constructive axiomatic}

  The axiomatic is in form of a module, so that it can be implemented, either with the standard library or with axiom-free Coq terms.

\subsection{Ordered ring}

  The first set of axioms describes the ring operators. $\R$ is a parameter of sort $\Type$, 0 and 1 are two objects of $\R$ and +, *, - are operators on $\R$ respectively of arities 2, 2, 1.

  We also need an equality predicate, but we will define the strict order relation first and the other one will be defined thanks to this one. Indeed a proof of $x<y$ will contain a computational witness of an $\varepsilon$ such that $x+\varepsilon≤y$ so that building an inverse of $x$ knowing that $x\neq 0$ will be easier. The sort of the predicate $<$ is not important, as you can use an argument like ``constructive epsilon'' of the Coq'Art\cite{coqart}\footnote{This trick is also in the standard library of Coq in \href{http://coq.inria.fr/stdlib/Coq.Logic.ConstructiveEpsilon.html}{Logic/ConstructiveEpsilon}} book to constructively extract such an $\varepsilon$. The axioms on $<$ are asymmetry and transitivity. We then define the relation of discriminability $\#$ as the constructive sum of $<$ and $<$, the inverse relation of $<$. $≡$ is the negation of $\#$ and $≤$ is the constructive sum of $<$ and $≡$.



\subsection{Constructive field}

  Now we can define an operator for the inverse, which expect a real number and the
  proof that this number is discriminable from 0:

  \[
    \cdot^{-1}_{[\cdot]} \ : \  (x:\R), (p:x\#0) \mapsto x^{-1}_{[p]} : \R
  \]

  We then add the axioms of commutativity, associativity, distributivity of multiplication and/over
  addition with respect to $≡$, and the axioms specifying 0 and 1 as the unit for $+$ and $*$. We also specify the inverse operators $-$, $^{-1}$ for $+$, $*$, the latter requiring a proof of $x\#0$.

  Moreover we have to say that $+$, $*$, $^{-1}$ and even $≡$ behave well with respect to $<$ and $≡$:

  \[
    \begin{array}{rcl}
      x ≡ x' \;,\; x < y   &  \Rightarrow  & x' < y \\
      x ≡ x' \;,\; y < x   &  \Rightarrow  & y < x' \\
      y ≡ y'               &  \Rightarrow  & x + y ≡ x + y' \\
      y < y'                    &  \Rightarrow  & x + y < x + y' \\
      0 < x \;,\; y < y'        &  \Rightarrow  & x \cdot y < x \cdot y' \\
      0 < x \;,\; y ≡ y'   &  \Rightarrow  & x \cdot y ≡ x \cdot y' \\
      0<x \;,\;  p':0\#x      &  \Rightarrow  & 0 < x^{-1}_{[p']} \\
    \end{array}
  \]

  We also require that $0<1$ (it is equivalent to $0\#1$, assuming the other axioms)

  Then we need the fact that $\R$ is archimedean. This is not constructively equivalent to having powerful tools like the integer part function, as it is not continuous thus not constructively definable. But we can get back from the infinity of $\R$ to finitely representable objects as elements of $\Z$. In fact we can constructively have a function $\llcorner \cdot \lrcorner$ taking the representation of a real number $x$ and returning some integer $z$ such that\footnote{This condition can be replaced by any condition of the form $| x-\underline {\llcorner x \lrcorner} | < 1/2+\varepsilon$ where $\varepsilon>0$} $| x-\underline z | < 1$. We cannot constructively make $\llcorner \cdot \lrcorner$ total and extensional (i.e. $x≡y → \llcorner x \lrcorner=\llcorner y \lrcorner$) because this would make it discontinuous.

  Finally we complete $\R$ by adding all the limits of the Cauchy sequences of elements of $\R$.

  \[
    \begin{array}{rcl}
      \mbox{cauchy}(u) & := & \forall \varepsilon>0, \exists N, \forall p,q≥N \; |u_p - u_q | < \varepsilon \\
      u→l & := & \forall \varepsilon>0, \exists N, \forall n≥N, \; |u_n - l | < \varepsilon \\
    \end{array}\\
  \]

  And the axiom we add is:
  \[
    \mbox{cauchy}(u) \Rightarrow \exists l, \; u→l  
  \]


  With these axioms we hope to capture the constructive real numbers. For instance they do not allow to prove the Markov's principle\footnote{Markov's principle: if $P$ is decidable then $¬¬ (\exists n\ P n) \Rightarrow \exists n\ P n$} (semi-constructive \cite{Herbelin10}), the countable principle of omniscience\footnote{Countable principle of omniscience: if $P$ is decidable then the property $(\forall n\ P n)$ is decidable} (classical) or the decidability of equality on real numbers (co-semidecidable) all implied by the axioms of the standard library.


\section{Constructive/classical distinction}

  Dealing with real analysis often needs using non constructive axioms like the excluded middle or the axiom of choice. However we would like to use extraction of constructive proofs and to know whether the proof of a statement is constructive.

  Therefore, if we want to reason constructively we need to have predicates -- like the convergence of a sequence or the differentiability of a function -- defined in $\Type$. To reason classically we need predicates to be in $\Prop$. One solution to this problem is to duplicate the definitions of the predicates: one for computational content in $\Type$ and another for non-constructive statements in $\Prop$. Another is the use of a monad to embed constructive predicates into a world where the statements are weaker.


\subsection{Dealing with axioms}

  Monads permits to keep track of the external hypothesis used in a proof, through the so-called \emph{axiom} monad.

  \begin{definition}
    The axiom monad on $X$ is $T A = X \rightarrow A$ with weakening as $\mathtt{return}$ and contraction as $\mathtt{bind}$.
  \end{definition}

  Interesting instantiations for $X$ include excluded-middle, choice, epsilon and virtually any axiom independent from pCIC.

  This monad permits to have a statically checked analysis of the requirements of a proof. One can also imagine to use a \emph{generic} axiom monad, under the form $T X A$ where $X$ is the list of axioms used.

  \begin{remark}
    The easy use of the axiom monads requires to assume the functional extensionality, as most of the time we want to prove properties such as $\mathtt{lift}\ x = \mathtt{lift}\ y$.
  \end{remark} %% WTF?

  Sometimes, additional axioms are not that terrible, and one can argue about the burden of the axiom monad. For instance, excluded middle or propositional extensionality are not havoc-wreaking enough to justify embedding them in a monad.

  Following Castéran's work \cite{casteranepsilon}, non-constructivism can provide an interesting use of the axiom monad. We recall the \emph{epsilon} axiom:

$$\varepsilon ≡ \forall A.\forall (P : A\rightarrow  \mathtt{Prop}).\ (\mathtt{exists}\ x, P x) \rightarrow \{x\mid P\  x\}$$

  The epsilon axiom is problematic because once it has been admitted, one cannot differentiate between proofs that are extractible and those which are not. Indeed, it allows informative content flow between $\mathtt{Prop}$ and $\mathtt{Type}$.

  Using the axiom monad on epsilon gives us back the static discrimination between constructive and non-constructive proofs.

\subsection{$\mathtt{Prop}$-$\mathtt{Type}$ distinction}

  A weaker way to encode epsilon is to embed every type in $\mathtt{Prop}$. This is done through the inhabited monad.
  $$\mathtt{Inductive\ inhabited\ (A : Type) : Prop :=  inhabits : A \rightarrow inhabited\ A}$$

  %We can recover an ersatz of the axiom monad using the choice axiom. REWRITTEN INTO:
  
  If we consider using the choice axiom\footnote{The choice axiom, which is purely non-informative:\\$\forall (A:\Type)\ (B:A→\Type), (\forall x:A,\ \mbox{inhabited}(B x)) \Rightarrow \mbox{inhabited}(\forall x:A, B~x)$}, we can recover a equivalent system than through the use of epsilon axiom monad.
  
  %%TODO: voir si c'est équivalent

  One needs to take care of this $\mathtt{inhabited}$ construction. Indeed, whenever proof-irrelevance is assumed, there is at most one proof of $\mathtt{inhabited}\ A$ for any $A$. This implies in particular that the programmer must fully specify the content of the lift thru $\mathtt{sig}$-like dependent pairs.

  For example with the choice axiom, we can specify the partial function inverse on real numbers as follows:
  $$\mathtt{inhabited}\ \{f\ :\ \R → \R\ |\ \forall x\ x \# 0 → f x * x ≡ 1\}$$

  \bibliographystyle{alpha}
  \bibliography{ref}

\end{document}
