\documentclass[draft]{beamer}
\usepackage[utf8x]{inputenc}  %
\usetheme[titlepagelogo=coqtail_big,
          alternativetitlepage=true,% Use the fancy title page.
          ]{Torino}
\usecolortheme{coqtail}

\author{Jean-Marie Madiot, Pierre-Marie Pèdrot}
\title{Constructive axiomatic for the real numbers}
\institute{Coqtail Junior Laboratory\\
	ENS Lyon}
\date{August, 26th}


\newcommand{\R}{\mathbb{R}}
\newcommand{\Q}{\mathbb{Q}}
\newcommand{\Z}{\mathbb{Z}}
\newcommand{\N}{\mathbb{N}}
\newcommand{\Type}{\mbox{Type}}
\newcommand{\Prop}{\mbox{Prop}}
\newcommand{\coqtail}{\textsc{coqtail}}

\begin{document}

\begin{frame}[t,plain]
\titlepage
\end{frame}

\begin{frame}{Related works}
  \begin{itemize}
    \item Coq's stdlib: highly classical (WEM in Set), 1/0
    \item C-CoRN: too complex, quite old, too intuitionistic
  \end{itemize}
  $→$ we need an interface!
  % Canonical structures = vieilles typeclasses
\end{frame}

\begin{frame}{Axioms}
  \begin{itemize}
    \item Ordered ring
      \begin{itemize}
        \item $(\R, ≤)$ is a preorder
        \item $(\R, +, 0, *, 1)$ is a ring
      \end{itemize}
    \item Constructive field
  \end{itemize}
\end{frame}

\begin{frame}{Constructive/classical distinction}
  \begin{itemize}
    \item axiom monad (functional extensionality, Castéran's $\varepsilon$)
    \item Prop-Type distinction (inhabited monad)
  \end{itemize}
\end{frame}



\end{document}

