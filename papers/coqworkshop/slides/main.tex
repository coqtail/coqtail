\documentclass{beamer}
\usepackage[utf8x]{inputenc}  %
\usetheme[titlepagelogo=coqtail_big,
          alternativetitlepage=true,% Use the fancy title page.
          ]{Torino}
\usecolortheme{coqtail}

\author{Jean-Marie Madiot, Pierre-Marie Pédrot}
\title{Constructive axiomatic for the real numbers}
\institute{Coqtail Junior Laboratory\\
	ENS Lyon}
\date{August, 26th}


\newcommand{\R}{\mathbb{R}}
\newcommand{\Q}{\mathbb{Q}}
\newcommand{\Z}{\mathbb{Z}}
\newcommand{\N}{\mathbb{N}}
\newcommand{\Type}{\mbox{Type}}
\newcommand{\Prop}{\mbox{Prop}}
\newcommand{\coqtail}{\textsc{coqtail}}

\begin{document}

\begin{frame}[t,plain]
\titlepage
\end{frame}

\begin{frame}{Real analysis in Coq}

\begin{itemize}
  \item Two major libraries of real analysis out there:
  \begin{itemize}
    \item Coq \texttt{stdlib}: highly classical
    \item C-CoRN: designed to be constructive
  \end{itemize}

  \item Libraries to prove analysis theorems
  \item There already have been attempts to mix both
  \pause

  \item Some other effective implementations, generally from the folks in computer arithmetic:
  \begin{itemize}
    \item In general, not libraries of theorems
    \item More about proved computation on real numbers
  \end{itemize}

\end{itemize}

\end{frame}

\begin{frame}{Coq \texttt{stdlib}}

\begin{itemize}
  \item \color{blue}{Pros}
  \begin{itemize}
    \item Simple and abstract formalism
    \item Designed with speed of development on mind
    \item Usual proofs are easy to write
    \item Shipped with Coq $\rightsquigarrow$ important base of users
  \end{itemize}\pause
  \item \color{red}{Cons}
  \begin{itemize}
    \item lacks a lot of basic results (on sequences, etc.)
    \item names quite messy: lemmas \texttt{Riemann\_tech24} and alike
    \item globally badly designed library ($\mathtt{Ranalysis}_i$ for $0 \le i \le 4$)
    \item worse than \alert{highly} classical:
    $$\forall A : \mathtt{Prop}, \{\neg\neg A\} + \{\neg A\}$$
    $$\{\forall n, P n\} + \{\exists n,\neg (P n)\}\quad \text{if $P : \mathtt{nat}\rightarrow\mathtt{Prop}$ decidable}$$
  \end{itemize}

\end{itemize}

\end{frame}

\begin{frame}{C-CoRN}

\begin{itemize}
  \item \color{blue}{Pros}
  \begin{itemize}
    \item Constructive: compute with your proofs!
    \item Huge database of intermediate structures
    \item A lot of nice results
  \end{itemize}\pause
  \item \color{red}{Cons}
  \begin{itemize}
    \item Constructive: make mathematicians flee away!
    \begin{itemize}
      \item[$\hookrightarrow$] ``we want real maths!''
    \end{itemize}
    \item Too complicated to use (from compilation to dependency hell)
    \item Oldish and overly module-relying Coq
    \item Not really maintained anymore
  \end{itemize}
\end{itemize}


\end{frame}

\begin{frame}{Manifesto}

We want to keep the best of both worlds:

\begin{itemize}
  \item From \texttt{stdlib}:
  \begin{itemize}
    \item Simplicity and abstraction
    \begin{itemize}
      \item[$\hookrightarrow$] that means fresh axiomatic construction
    \end{itemize}
    \item Ability to do classical proofs
    \begin{itemize}
      \item[$\hookrightarrow$] fancy axioms in \texttt{Prop} accepted
      \item[$\hookrightarrow$] excluded middle, choice axiom...
      \item[$\hookrightarrow$] may port classical tactics (\texttt{fourier}, \texttt{psatz}...)
    \end{itemize}
  \end{itemize}\pause
  \item From C-CoRN:
  \begin{itemize}
    \item Constructive axiomatic
    \begin{itemize}
      \item[$\hookrightarrow$] results in \texttt{Type} extractible
      \item[$\hookrightarrow$] plugging in C-CoRN?
    \end{itemize}
    \item \emph{Tabula rasa} $\rightsquigarrow$ lot of work
  \end{itemize}
\end{itemize}\pause

Neither obvious nor easy!

\end{frame}

\begin{frame}{Caveats}

A fundamental problem: \alert{equality}!

\begin{itemize}
  \item Leibniz equality on real numbers is a quotient
  \begin{itemize}
    \item $x \simeq y := \forall \varepsilon > 0, \delta(x, y) < \varepsilon $
  \end{itemize}
  \item Whenever we can approximate $\mathbb{R}$ this is not constructive
    \begin{itemize}
      \item ... as $\mathtt{eval}_\varepsilon : \mathbb{R}\rightarrow\mathbb{Q}$ is not continuous
      \item while any field operation must be compatible with $\simeq$
      \item $\mathtt{eval}_\varepsilon$ cannot be
      \begin{itemize}
	\item[$\hookrightarrow$] this would break extraction
      \end{itemize}
    \end{itemize}\pause
  \item Really need to use setoids
\end{itemize}

Other problems related to constructivity: \alert{partial} functions must take a proof, and other non-equivalences

\end{frame}

\begin{frame}{Actual Axiomatic}

\begin{itemize}
  \item Structure and operations:
\begin{itemize}
  \item axiomatic: $\mathbb{R}:\mathtt{Type}$, $< : \mathbb{R}\rightarrow\mathbb{R}\rightarrow\mathtt{Prop}$ 
  \item derived:
  \begin{align*}
  x\gtrless y &:= \{x < y\} + \{y < x\} & \text{in \texttt{Type}}\\
  x \simeq y &:= \neg (x < y) \wedge \neg(y < x) &\text{in \texttt{Prop}}
  \end{align*}

  \item axiomatic: $+$, $-$, $\times$, $(\cdot)^{-1}$ (with a proof that $x\gtrless 0$)
\end{itemize}

  \item Usual well-behavedness axioms

  \item Important issue: completeness!
  \begin{itemize}
  \item As in C-CoRN
  \item Constructive Cauchy sequences (\texttt{sig} in \texttt{Type})
  \item Any such sequence converges
  \end{itemize}

\end{itemize}

\end{frame}

\begin{frame}{EM and Markov's principle}

Assuming EM in \texttt{Prop}, we get Markov's principle for free.

  \begin{itemize}
  \item Equivalent to $\neg\neg x > 0 \rightarrow x > 0$
  \begin{itemize}
    \item which already implied $\{n :\mathbb{N} \mid x > 2^{-n} \}$
  \end{itemize}
  \item Simplifies a lot of reasonings
  \begin{itemize}
    \item we could implement Fourier elimination
  \end{itemize}
  \item We can still kind of compute using tactic-based \texttt{Acc} trick
  \end{itemize}

EM is not harmful for constructivity: things in \texttt{Prop} were considered irrelevant from the beginning.

\end{frame}

\begin{frame}{Really Classical Maths}

\begin{itemize}
  \item Up to now, we're not a lot different from a slightly classical C-CoRN
  \item Castéran proposal: a violent axiom that confuse \texttt{Prop} and \texttt{Type}
  {\small $$\varepsilon:\forall A\ P, (\texttt{exists}\ x, P\,x) \rightarrow \{x\mid P\,x\} $$}
  \item We do not want to mix non-constructive results with constructive ones
  \begin{itemize}
    \item ... there is \texttt{stdlib} for this!
  \end{itemize}
  \item We use a structure inherited from programming: \alert{monads}!
  \begin{itemize}
    \item A type constructor $T : \texttt{Type}\rightarrow\texttt{Type}$
    \item A lift $A \rightarrow T A$ and a join $T^2 A \rightarrow A$
    \item in general, no information flow $T A \rightarrow A$
    \begin{itemize}
      \item[$\rightsquigarrow$] that's what we wanted
    \end{itemize}
  \end{itemize}
\end{itemize}

\end{frame}

\begin{frame}{The Inhabited Monad}

The inhabited monad : a singleton constructor in \texttt{Prop}
{\small$$\mathtt{Inductive\ inhabited}\ A : \mathtt{Prop} := \mathtt{inhabits} : A \rightarrow \mathtt{inhabited}\,A.$$}
\pause
\begin{itemize}
  \item turn any constructive predicate in a non-constructive one in \texttt{Prop}
  \item quite delicate to use
  \begin{itemize}
    \item need explicit specifications (proof-irrelevance!)
  \end{itemize}
  \item actually this is really useful together with the full axiom of choice
{\small$$\forall A\ (B : A \rightarrow \mathtt{Type}), (\forall x, \mathtt{inhabited}\ (B\,x)) \rightarrow
\mathtt{inhabited}\ (\forall x, B\,x)$$}
\end{itemize}

\end{frame}

\begin{frame}{The Axiom Monad}

A very generic and useful monad.

\begin{itemize}
  \item The axiom monad : $T_X A = X \rightarrow A$
  \item used with $X = \varepsilon$ provides the full power of choice axiom
  \begin{itemize}
    \item partial functions, elimination of dependence in proofs and much more!
    \item easy to use in a mathematical fashion
    \item equivalent to \texttt{inhabited} + AC
  \end{itemize}
    \item can be refined with any other powerful axiom at will

\end{itemize}

We can virtually tag any result with its degree of classicality (here we're just interessed in algorithmic realizability).

\end{frame}

\begin{frame}{Conclusion}

\begin{itemize}
  \item Writing a manageable real arithmetic library is difficult
  \begin{itemize}
    \item on a theoretical point of view: carefully choose your system!
    \item on a practical point of view: naming conventions and usability
  \end{itemize}
  \item Rewriting a whole generic library from scratch is a tedious work
  \item Actually we don't have any interesting result yet...
  \begin{itemize}
    \item just a bunch of basic lemmas wich are the most cumbersome to write
    \item we badly lack tactics for now
  \end{itemize}
  \item Using monads to discriminate proof properties seems something new
\end{itemize}

\end{frame}

\begin{frame}{Scribitur ad narrandum, not ad probandum}

\begin{center}
{\Huge Thank you.}
\end{center}

\end{frame}

\end{document}

